% Supplemental material extracted from `docs/catalan-light-cone.tex`.
% This block was removed from the compiled arXiv PDF to keep the main v1 lean.
% Intended use: `\% Supplemental material extracted from `docs/catalan-light-cone.tex`.
% This block was removed from the compiled arXiv PDF to keep the main v1 lean.
% Intended use: `\% Supplemental material extracted from `docs/catalan-light-cone.tex`.
% This block was removed from the compiled arXiv PDF to keep the main v1 lean.
% Intended use: `\% Supplemental material extracted from `docs/catalan-light-cone.tex`.
% This block was removed from the compiled arXiv PDF to keep the main v1 lean.
% Intended use: `\\input{supplemental-operators.tex}` at the former location in
% Appendix "Additional Technical Notes".

\subsection{Fields on words, prefixes, and nodes (optional)}
\label{subsec:fields}

We use the word ``field'' as shorthand for a complex-valued function on one of
the Catalan objects already in play. Several closely related state spaces are
useful in different contexts.

\paragraph{Fields on completed histories (fixed tier).}
Fix $n$ and consider a function $\Phi_n:\mathcal D_n\to\mathbb{C}$ assigning an
amplitude (or observable value) to each completed history $w\in\mathcal D_n$.
The associated Hilbert space is $\ell^2(\mathcal D_n)$ with inner product
\[
\langle \psi,\phi\rangle := \sum_{w\in\mathcal D_n} \overline{\psi(w)}\,\phi(w).
\]

\paragraph{Fields on prefixes (the full cone).}
Let $\mathcal{C}$ denote the set of Dyck prefixes (admissible partial histories).
A prefix field is a function $\Phi:\mathcal{C}\to\mathbb{C}$, which may be
restricted to a fixed length slice
$\mathcal{C}^{(k)}:=\{p\in\mathcal{C}: |p|=k\}$ when needed.

\paragraph{Fields on nodes of a fixed tree.}
Given $w\in\mathcal D_n$, let $T(w)$ be its associated full binary tree. A
node field is a function $\phi_w:\mathrm{Int}(T(w))\to\mathbb{C}$ on the internal
nodes of that tree.

\begin{remark}
These notions live on different objects (tiers, the prefix poset, or a single
tree) and are independent of any within-tier ordering convention on
$\mathcal D_n$.
\end{remark}

\subsection{Subtree indicators as a multiscale spanning family (optional)}
\label{subsec:subtree-indicators}

Let $T$ be a finite rooted tree and write $\mathrm{Int}(T)$ for its internal
nodes. Each $v\in\mathrm{Int}(T)$ determines a rooted subtree $T_v$, and hence a
subset $\mathrm{Int}(T_v)\subseteq \mathrm{Int}(T)$. Define the subtree indicator
\[
\chi_v:\mathrm{Int}(T)\to\{0,1\},
\qquad
\chi_v(u):=\mathbf{1}\{u\in \mathrm{Int}(T_v)\}.
\]

\begin{lemma}[Subtree indicators form a basis]
\label{lem:subtree-indicator-basis}
The family $\{\chi_v: v\in \mathrm{Int}(T)\}$ is a basis of the vector space of
complex-valued functions on $\mathrm{Int}(T)$.
\end{lemma}

\begin{proof}
Order the internal nodes by nonincreasing depth (deepest first), and let $M$ be
the square matrix with entries $M_{uv}:=\chi_v(u)$. Then $M_{vv}=1$ for all $v$,
while $M_{uv}=0$ whenever $u$ precedes $v$ in this order (a node cannot be a
descendant of a deeper node). Thus $M$ is triangular with ones on the diagonal,
hence invertible. Therefore the indicators are linearly independent and, since
their number equals $\#\mathrm{Int}(T)$, they form a basis.
\end{proof}

\begin{corollary}[Explicit inversion]
\label{cor:subtree-indicator-inversion}
Let $f:\mathrm{Int}(T)\to\mathbb{C}$ be any function. There is a unique family
of coefficients $\{a_v\}_{v\in\mathrm{Int}(T)}$ such that
\[
f \;=\; \sum_{v\in\mathrm{Int}(T)} a_v\,\chi_v.
\]
Writing $\mathrm{par}(v)$ for the parent of $v$ (for $v\neq \mathrm{root}(T)$),
these coefficients are given by
\[
a_{\mathrm{root}(T)} = f(\mathrm{root}(T)),
\qquad
a_v = f(v)-f(\mathrm{par}(v)) \quad (v\neq \mathrm{root}(T)).
\]
\end{corollary}

\begin{proof}
For each $u\in\mathrm{Int}(T)$,
$(\sum_v a_v\chi_v)(u)=\sum_{v:\,u\in\mathrm{Int}(T_v)} a_v
=\sum_{v\preceq u} a_v$, where $v\preceq u$ means that $v$ is an ancestor of
$u$. With the stated choice of coefficients, this ancestor sum telescopes along
the unique root-to-$u$ chain to yield $f(u)$. Uniqueness follows from
Lemma~\ref{lem:subtree-indicator-basis}.
\end{proof}

\begin{remark}
This basis is ``multiscale'': indicators of deep subtrees localize to fine
regions of $T$, while indicators near the root encode coarse structure. Any
choice of orthonormalization yields an orthonormal basis adapted to the rooted
tree geometry.
\end{remark}

\subsection{Operators on a fixed history tree (optional)}
\label{subsec:tree-operators}

In addition to tier-wise state spaces (fields on $\mathcal D_n$), one may also
consider dynamics \emph{within} a fixed realized history by placing operators on
the internal nodes of its tree.

\paragraph{Node Hilbert space.}
Fix $w\in\mathcal D_n$ and let $T(w)$ be its associated full binary tree. Write
$V_w:=\mathrm{Int}(T(w))$ and consider $\ell^2(V_w)$ with inner product
$\langle \psi,\phi\rangle := \sum_{v\in V_w}\overline{\psi(v)}\,\phi(v)$.

\paragraph{Adjacency and Laplacian.}
Let $G_w=(V_w,E_w)$ be any finite undirected graph on $V_w$ (for example, connect
each internal node to its internal children). Define $A_{G_w}$, $D_{G_w}$, and
the graph Laplacian and generator by
\[
\Delta_{G_w}:=D_{G_w}-A_{G_w},
\qquad
L_{G_w}:=-\Delta_{G_w}.
\]

\paragraph{Heat and Schr\"odinger evolutions.}
The corresponding ``internal-time'' heat equation is
\[
\partial_\tau u = L_{G_w}u,
\]
and the corresponding unitary Schr\"odinger evolution is
\[
i\,\partial_t \psi = -L_{G_w}\psi = \Delta_{G_w}\psi.
\]

\begin{remark}
This within-history operator framework is independent of the tier-growth Markov
dynamics and of coherent summation over histories: it simply records that, once
a graph structure is specified on the internal nodes of a fixed Catalan tree,
standard graph-Laplacian constructions yield discrete diffusion and
Schr\"odinger-type evolutions on that fixed combinatorial background.
\end{remark}

\subsection{Operators on tier slices (optional)}
\label{subsec:tier-operators}

The main text emphasizes two dynamics on the Catalan substrate: tier growth
(prefix extension) and coherent summation over histories. Independently, one may
also consider \emph{slice dynamics} on a fixed tier by endowing the finite set
$\mathcal D_n$ with an auxiliary adjacency graph. This subsection records the
standard operator framework for such constructions.

\paragraph{Tier Hilbert space.}
We take the tier state space to be $\ell^2(\mathcal D_n)$ as in
Section~\ref{subsec:fields}.

\paragraph{Adjacency graphs.}
Let $G_n=(\mathcal D_n,E_n)$ be any finite undirected graph on $\mathcal D_n$.
The choice of $G_n$ is additional structure: different graphs induce different
notions of locality on the tier. A canonical example is the rotation graph (the
associahedron adjacency) on full binary trees, where edges correspond to single
associativity rotations \cite{stanley-catalan,cheneviere2022linear}.

\paragraph{Associahedra and planar tree amplitudes (scattering-amplitude tie-in).}
The associahedron adjacency on $\mathcal D_n$ is also natural from the
perspective of scattering amplitudes. For the planar tree-level sector of
bi-adjoint cubic scalar theory ($\phi^3$), Arkani-Hamed, Bai, He, and Yan
identify an associahedron in planar kinematic space and show that the tree
amplitude is the corresponding canonical form of this positive geometry
\cite{abhy2018scatteringforms}. From this viewpoint, the Catalan enumeration of
planar cubic tree diagrams is not merely counting: the associahedron organizes
factorization channels geometrically, and different triangulations correspond to
different diagrammatic expansions of the same canonical form (see, e.g., the
review \cite{herrmann2022positivegeometry}). For quartic interactions, an
analogous positive-geometry description involves Stokes polytopes rather than
associahedra \cite{banerjee2018stokes}.

\paragraph{Tamari/Dyck/alt-Tamari choices on the same tier.}
The point of introducing an auxiliary graph $G_n$ is that the underlying state
set $\mathcal D_n$ supports multiple natural notions of tier-locality coming
from classical Catalan posets. The rotation graph is the undirected adjacency
underlying the Tamari order; one may likewise equip $\mathcal D_n$ with
adjacency induced by the Dyck (``Stanley'') lattice on Dyck paths, or more
generally by the family of $\delta$-Tamari (alt-Tamari) posets interpolating
between these extremes. These alternatives use different covering relations on
the same Catalan tier and therefore induce different graph Laplacians
$\Delta_{G_n}$ and different ``free'' tier Hamiltonians, but they live on a
common configuration space $\mathcal D_n$ \cite{stanley-catalan,cheneviere2022linear}.

\paragraph{Linear intervals as ``1D corridors'' in Catalan posets.}
A useful robustness fact is that certain one-dimensional substructures are
invariant across these Catalan posets: Chenevi\`ere proves that, for each fixed
$n$ and each height parameter $k$ (in the sense of \cite{cheneviere2022linear}),
the Tamari lattice and the Dyck lattice have the same number of \emph{linear
intervals} (intervals whose Hasse diagram is a chain), and moreover all
alt-Tamari posets share this same count at each height
\cite{cheneviere2022linear}. In the present language, this says that the number
of ``diamond-free corridors'' (regions with a unique maximal chain) is stable
under a wide class of tier-local adjacency choices, reinforcing the theme that
many distinct dynamics can be layered on a single Catalan substrate without
changing its most rigid combinatorial invariants.

\paragraph{Adjacency and Laplacian.}
Define the adjacency operator $A_{G_n}$ and the degree operator $D_{G_n}$ on
$\ell^2(\mathcal D_n)$ by
\[
(A_{G_n}\psi)(w) := \sum_{w'\sim w} \psi(w'),
\qquad
(D_{G_n}\psi)(w) := \deg(w)\,\psi(w),
\]
where $w'\sim w$ denotes adjacency in $G_n$. The (combinatorial) graph Laplacian is
\[
\Delta_{G_n} := D_{G_n}-A_{G_n},
\]
and the associated diffusion generator is
\[
L_{G_n} := -\Delta_{G_n}.
\]
Then $\Delta_{G_n}$ is self-adjoint and positive semidefinite, while $L_{G_n}$ is
self-adjoint and negative semidefinite.

\paragraph{Discrete heat and Schr\"odinger equations.}
The heat equation on the tier graph is the linear ODE
\[
\partial_\tau u = L_{G_n}u,
\]
with solution $u(\tau)=e^{\tau L_{G_n}}u(0)$. Because $-\Delta_{G_n}$ is
self-adjoint and nonpositive, $e^{\tau L_{G_n}}$ is a contraction semigroup.
The corresponding unitary ``free'' Schr\"odinger evolution is
\[
i\,\partial_t \psi = -L_{G_n}\psi = \Delta_{G_n}\psi,
\]
with solution $\psi(t)=e^{-it\Delta_{G_n}}\psi(0)$.

\begin{remark}
This optional tier-graph framework does not fix a preferred choice of adjacency
$G_n$ and is not used elsewhere in the paper. Its purpose is to make explicit
that, once a tier-local notion of neighbourhood is specified, discrete diffusion
and Schr\"odinger-type evolutions on the Catalan state space follow by standard
graph-Laplacian constructions (compare Remark~\ref{rem:dyck-conditioned-drift-scaling}
and Remark~\ref{rem:dyck-conditioned-kernel-doob} for the tier-growth Markov
structure induced by Dyck conditioning).
\end{remark}

% ---------------------------------------------------------------------------
% Bibitems used only in this supplement (kept here for copy/paste convenience).
% If you re-`\input{supplemental-operators.tex}` into the main paper, these
% entries should live in the main `thebibliography` environment.
% ---------------------------------------------------------------------------
\iffalse
\begin{thebibliography}{99}

\bibitem{abhy2018scatteringforms}
N.~Arkani-Hamed, Y.~Bai, S.~He, and G.~Yan.
\newblock Scattering forms and the positive geometry of kinematics, color and
the worldsheet.
\newblock {\em JHEP} \textbf{05} (2018) 096.
\newblock arXiv:1711.09102.

\bibitem{banerjee2018stokes}
P.~Banerjee, A.~Laddha, and P.~Raman.
\newblock Stokes polytopes: The positive geometry for $\phi^4$ interactions.
\newblock arXiv:1811.05904, 2018.

\bibitem{herrmann2022positivegeometry}
E.~Herrmann and J.~Trnka.
\newblock Positive geometry of scattering amplitudes.
\newblock arXiv:2203.13018, 2022.

\end{thebibliography}
\fi
` at the former location in
% Appendix "Additional Technical Notes".

\subsection{Fields on words, prefixes, and nodes (optional)}
\label{subsec:fields}

We use the word ``field'' as shorthand for a complex-valued function on one of
the Catalan objects already in play. Several closely related state spaces are
useful in different contexts.

\paragraph{Fields on completed histories (fixed tier).}
Fix $n$ and consider a function $\Phi_n:\mathcal D_n\to\mathbb{C}$ assigning an
amplitude (or observable value) to each completed history $w\in\mathcal D_n$.
The associated Hilbert space is $\ell^2(\mathcal D_n)$ with inner product
\[
\langle \psi,\phi\rangle := \sum_{w\in\mathcal D_n} \overline{\psi(w)}\,\phi(w).
\]

\paragraph{Fields on prefixes (the full cone).}
Let $\mathcal{C}$ denote the set of Dyck prefixes (admissible partial histories).
A prefix field is a function $\Phi:\mathcal{C}\to\mathbb{C}$, which may be
restricted to a fixed length slice
$\mathcal{C}^{(k)}:=\{p\in\mathcal{C}: |p|=k\}$ when needed.

\paragraph{Fields on nodes of a fixed tree.}
Given $w\in\mathcal D_n$, let $T(w)$ be its associated full binary tree. A
node field is a function $\phi_w:\mathrm{Int}(T(w))\to\mathbb{C}$ on the internal
nodes of that tree.

\begin{remark}
These notions live on different objects (tiers, the prefix poset, or a single
tree) and are independent of any within-tier ordering convention on
$\mathcal D_n$.
\end{remark}

\subsection{Subtree indicators as a multiscale spanning family (optional)}
\label{subsec:subtree-indicators}

Let $T$ be a finite rooted tree and write $\mathrm{Int}(T)$ for its internal
nodes. Each $v\in\mathrm{Int}(T)$ determines a rooted subtree $T_v$, and hence a
subset $\mathrm{Int}(T_v)\subseteq \mathrm{Int}(T)$. Define the subtree indicator
\[
\chi_v:\mathrm{Int}(T)\to\{0,1\},
\qquad
\chi_v(u):=\mathbf{1}\{u\in \mathrm{Int}(T_v)\}.
\]

\begin{lemma}[Subtree indicators form a basis]
\label{lem:subtree-indicator-basis}
The family $\{\chi_v: v\in \mathrm{Int}(T)\}$ is a basis of the vector space of
complex-valued functions on $\mathrm{Int}(T)$.
\end{lemma}

\begin{proof}
Order the internal nodes by nonincreasing depth (deepest first), and let $M$ be
the square matrix with entries $M_{uv}:=\chi_v(u)$. Then $M_{vv}=1$ for all $v$,
while $M_{uv}=0$ whenever $u$ precedes $v$ in this order (a node cannot be a
descendant of a deeper node). Thus $M$ is triangular with ones on the diagonal,
hence invertible. Therefore the indicators are linearly independent and, since
their number equals $\#\mathrm{Int}(T)$, they form a basis.
\end{proof}

\begin{corollary}[Explicit inversion]
\label{cor:subtree-indicator-inversion}
Let $f:\mathrm{Int}(T)\to\mathbb{C}$ be any function. There is a unique family
of coefficients $\{a_v\}_{v\in\mathrm{Int}(T)}$ such that
\[
f \;=\; \sum_{v\in\mathrm{Int}(T)} a_v\,\chi_v.
\]
Writing $\mathrm{par}(v)$ for the parent of $v$ (for $v\neq \mathrm{root}(T)$),
these coefficients are given by
\[
a_{\mathrm{root}(T)} = f(\mathrm{root}(T)),
\qquad
a_v = f(v)-f(\mathrm{par}(v)) \quad (v\neq \mathrm{root}(T)).
\]
\end{corollary}

\begin{proof}
For each $u\in\mathrm{Int}(T)$,
$(\sum_v a_v\chi_v)(u)=\sum_{v:\,u\in\mathrm{Int}(T_v)} a_v
=\sum_{v\preceq u} a_v$, where $v\preceq u$ means that $v$ is an ancestor of
$u$. With the stated choice of coefficients, this ancestor sum telescopes along
the unique root-to-$u$ chain to yield $f(u)$. Uniqueness follows from
Lemma~\ref{lem:subtree-indicator-basis}.
\end{proof}

\begin{remark}
This basis is ``multiscale'': indicators of deep subtrees localize to fine
regions of $T$, while indicators near the root encode coarse structure. Any
choice of orthonormalization yields an orthonormal basis adapted to the rooted
tree geometry.
\end{remark}

\subsection{Operators on a fixed history tree (optional)}
\label{subsec:tree-operators}

In addition to tier-wise state spaces (fields on $\mathcal D_n$), one may also
consider dynamics \emph{within} a fixed realized history by placing operators on
the internal nodes of its tree.

\paragraph{Node Hilbert space.}
Fix $w\in\mathcal D_n$ and let $T(w)$ be its associated full binary tree. Write
$V_w:=\mathrm{Int}(T(w))$ and consider $\ell^2(V_w)$ with inner product
$\langle \psi,\phi\rangle := \sum_{v\in V_w}\overline{\psi(v)}\,\phi(v)$.

\paragraph{Adjacency and Laplacian.}
Let $G_w=(V_w,E_w)$ be any finite undirected graph on $V_w$ (for example, connect
each internal node to its internal children). Define $A_{G_w}$, $D_{G_w}$, and
the graph Laplacian and generator by
\[
\Delta_{G_w}:=D_{G_w}-A_{G_w},
\qquad
L_{G_w}:=-\Delta_{G_w}.
\]

\paragraph{Heat and Schr\"odinger evolutions.}
The corresponding ``internal-time'' heat equation is
\[
\partial_\tau u = L_{G_w}u,
\]
and the corresponding unitary Schr\"odinger evolution is
\[
i\,\partial_t \psi = -L_{G_w}\psi = \Delta_{G_w}\psi.
\]

\begin{remark}
This within-history operator framework is independent of the tier-growth Markov
dynamics and of coherent summation over histories: it simply records that, once
a graph structure is specified on the internal nodes of a fixed Catalan tree,
standard graph-Laplacian constructions yield discrete diffusion and
Schr\"odinger-type evolutions on that fixed combinatorial background.
\end{remark}

\subsection{Operators on tier slices (optional)}
\label{subsec:tier-operators}

The main text emphasizes two dynamics on the Catalan substrate: tier growth
(prefix extension) and coherent summation over histories. Independently, one may
also consider \emph{slice dynamics} on a fixed tier by endowing the finite set
$\mathcal D_n$ with an auxiliary adjacency graph. This subsection records the
standard operator framework for such constructions.

\paragraph{Tier Hilbert space.}
We take the tier state space to be $\ell^2(\mathcal D_n)$ as in
Section~\ref{subsec:fields}.

\paragraph{Adjacency graphs.}
Let $G_n=(\mathcal D_n,E_n)$ be any finite undirected graph on $\mathcal D_n$.
The choice of $G_n$ is additional structure: different graphs induce different
notions of locality on the tier. A canonical example is the rotation graph (the
associahedron adjacency) on full binary trees, where edges correspond to single
associativity rotations \cite{stanley-catalan,cheneviere2022linear}.

\paragraph{Associahedra and planar tree amplitudes (scattering-amplitude tie-in).}
The associahedron adjacency on $\mathcal D_n$ is also natural from the
perspective of scattering amplitudes. For the planar tree-level sector of
bi-adjoint cubic scalar theory ($\phi^3$), Arkani-Hamed, Bai, He, and Yan
identify an associahedron in planar kinematic space and show that the tree
amplitude is the corresponding canonical form of this positive geometry
\cite{abhy2018scatteringforms}. From this viewpoint, the Catalan enumeration of
planar cubic tree diagrams is not merely counting: the associahedron organizes
factorization channels geometrically, and different triangulations correspond to
different diagrammatic expansions of the same canonical form (see, e.g., the
review \cite{herrmann2022positivegeometry}). For quartic interactions, an
analogous positive-geometry description involves Stokes polytopes rather than
associahedra \cite{banerjee2018stokes}.

\paragraph{Tamari/Dyck/alt-Tamari choices on the same tier.}
The point of introducing an auxiliary graph $G_n$ is that the underlying state
set $\mathcal D_n$ supports multiple natural notions of tier-locality coming
from classical Catalan posets. The rotation graph is the undirected adjacency
underlying the Tamari order; one may likewise equip $\mathcal D_n$ with
adjacency induced by the Dyck (``Stanley'') lattice on Dyck paths, or more
generally by the family of $\delta$-Tamari (alt-Tamari) posets interpolating
between these extremes. These alternatives use different covering relations on
the same Catalan tier and therefore induce different graph Laplacians
$\Delta_{G_n}$ and different ``free'' tier Hamiltonians, but they live on a
common configuration space $\mathcal D_n$ \cite{stanley-catalan,cheneviere2022linear}.

\paragraph{Linear intervals as ``1D corridors'' in Catalan posets.}
A useful robustness fact is that certain one-dimensional substructures are
invariant across these Catalan posets: Chenevi\`ere proves that, for each fixed
$n$ and each height parameter $k$ (in the sense of \cite{cheneviere2022linear}),
the Tamari lattice and the Dyck lattice have the same number of \emph{linear
intervals} (intervals whose Hasse diagram is a chain), and moreover all
alt-Tamari posets share this same count at each height
\cite{cheneviere2022linear}. In the present language, this says that the number
of ``diamond-free corridors'' (regions with a unique maximal chain) is stable
under a wide class of tier-local adjacency choices, reinforcing the theme that
many distinct dynamics can be layered on a single Catalan substrate without
changing its most rigid combinatorial invariants.

\paragraph{Adjacency and Laplacian.}
Define the adjacency operator $A_{G_n}$ and the degree operator $D_{G_n}$ on
$\ell^2(\mathcal D_n)$ by
\[
(A_{G_n}\psi)(w) := \sum_{w'\sim w} \psi(w'),
\qquad
(D_{G_n}\psi)(w) := \deg(w)\,\psi(w),
\]
where $w'\sim w$ denotes adjacency in $G_n$. The (combinatorial) graph Laplacian is
\[
\Delta_{G_n} := D_{G_n}-A_{G_n},
\]
and the associated diffusion generator is
\[
L_{G_n} := -\Delta_{G_n}.
\]
Then $\Delta_{G_n}$ is self-adjoint and positive semidefinite, while $L_{G_n}$ is
self-adjoint and negative semidefinite.

\paragraph{Discrete heat and Schr\"odinger equations.}
The heat equation on the tier graph is the linear ODE
\[
\partial_\tau u = L_{G_n}u,
\]
with solution $u(\tau)=e^{\tau L_{G_n}}u(0)$. Because $-\Delta_{G_n}$ is
self-adjoint and nonpositive, $e^{\tau L_{G_n}}$ is a contraction semigroup.
The corresponding unitary ``free'' Schr\"odinger evolution is
\[
i\,\partial_t \psi = -L_{G_n}\psi = \Delta_{G_n}\psi,
\]
with solution $\psi(t)=e^{-it\Delta_{G_n}}\psi(0)$.

\begin{remark}
This optional tier-graph framework does not fix a preferred choice of adjacency
$G_n$ and is not used elsewhere in the paper. Its purpose is to make explicit
that, once a tier-local notion of neighbourhood is specified, discrete diffusion
and Schr\"odinger-type evolutions on the Catalan state space follow by standard
graph-Laplacian constructions (compare Remark~\ref{rem:dyck-conditioned-drift-scaling}
and Remark~\ref{rem:dyck-conditioned-kernel-doob} for the tier-growth Markov
structure induced by Dyck conditioning).
\end{remark}

% ---------------------------------------------------------------------------
% Bibitems used only in this supplement (kept here for copy/paste convenience).
% If you re-`% Supplemental material extracted from `docs/catalan-light-cone.tex`.
% This block was removed from the compiled arXiv PDF to keep the main v1 lean.
% Intended use: `\\input{supplemental-operators.tex}` at the former location in
% Appendix "Additional Technical Notes".

\subsection{Fields on words, prefixes, and nodes (optional)}
\label{subsec:fields}

We use the word ``field'' as shorthand for a complex-valued function on one of
the Catalan objects already in play. Several closely related state spaces are
useful in different contexts.

\paragraph{Fields on completed histories (fixed tier).}
Fix $n$ and consider a function $\Phi_n:\mathcal D_n\to\mathbb{C}$ assigning an
amplitude (or observable value) to each completed history $w\in\mathcal D_n$.
The associated Hilbert space is $\ell^2(\mathcal D_n)$ with inner product
\[
\langle \psi,\phi\rangle := \sum_{w\in\mathcal D_n} \overline{\psi(w)}\,\phi(w).
\]

\paragraph{Fields on prefixes (the full cone).}
Let $\mathcal{C}$ denote the set of Dyck prefixes (admissible partial histories).
A prefix field is a function $\Phi:\mathcal{C}\to\mathbb{C}$, which may be
restricted to a fixed length slice
$\mathcal{C}^{(k)}:=\{p\in\mathcal{C}: |p|=k\}$ when needed.

\paragraph{Fields on nodes of a fixed tree.}
Given $w\in\mathcal D_n$, let $T(w)$ be its associated full binary tree. A
node field is a function $\phi_w:\mathrm{Int}(T(w))\to\mathbb{C}$ on the internal
nodes of that tree.

\begin{remark}
These notions live on different objects (tiers, the prefix poset, or a single
tree) and are independent of any within-tier ordering convention on
$\mathcal D_n$.
\end{remark}

\subsection{Subtree indicators as a multiscale spanning family (optional)}
\label{subsec:subtree-indicators}

Let $T$ be a finite rooted tree and write $\mathrm{Int}(T)$ for its internal
nodes. Each $v\in\mathrm{Int}(T)$ determines a rooted subtree $T_v$, and hence a
subset $\mathrm{Int}(T_v)\subseteq \mathrm{Int}(T)$. Define the subtree indicator
\[
\chi_v:\mathrm{Int}(T)\to\{0,1\},
\qquad
\chi_v(u):=\mathbf{1}\{u\in \mathrm{Int}(T_v)\}.
\]

\begin{lemma}[Subtree indicators form a basis]
\label{lem:subtree-indicator-basis}
The family $\{\chi_v: v\in \mathrm{Int}(T)\}$ is a basis of the vector space of
complex-valued functions on $\mathrm{Int}(T)$.
\end{lemma}

\begin{proof}
Order the internal nodes by nonincreasing depth (deepest first), and let $M$ be
the square matrix with entries $M_{uv}:=\chi_v(u)$. Then $M_{vv}=1$ for all $v$,
while $M_{uv}=0$ whenever $u$ precedes $v$ in this order (a node cannot be a
descendant of a deeper node). Thus $M$ is triangular with ones on the diagonal,
hence invertible. Therefore the indicators are linearly independent and, since
their number equals $\#\mathrm{Int}(T)$, they form a basis.
\end{proof}

\begin{corollary}[Explicit inversion]
\label{cor:subtree-indicator-inversion}
Let $f:\mathrm{Int}(T)\to\mathbb{C}$ be any function. There is a unique family
of coefficients $\{a_v\}_{v\in\mathrm{Int}(T)}$ such that
\[
f \;=\; \sum_{v\in\mathrm{Int}(T)} a_v\,\chi_v.
\]
Writing $\mathrm{par}(v)$ for the parent of $v$ (for $v\neq \mathrm{root}(T)$),
these coefficients are given by
\[
a_{\mathrm{root}(T)} = f(\mathrm{root}(T)),
\qquad
a_v = f(v)-f(\mathrm{par}(v)) \quad (v\neq \mathrm{root}(T)).
\]
\end{corollary}

\begin{proof}
For each $u\in\mathrm{Int}(T)$,
$(\sum_v a_v\chi_v)(u)=\sum_{v:\,u\in\mathrm{Int}(T_v)} a_v
=\sum_{v\preceq u} a_v$, where $v\preceq u$ means that $v$ is an ancestor of
$u$. With the stated choice of coefficients, this ancestor sum telescopes along
the unique root-to-$u$ chain to yield $f(u)$. Uniqueness follows from
Lemma~\ref{lem:subtree-indicator-basis}.
\end{proof}

\begin{remark}
This basis is ``multiscale'': indicators of deep subtrees localize to fine
regions of $T$, while indicators near the root encode coarse structure. Any
choice of orthonormalization yields an orthonormal basis adapted to the rooted
tree geometry.
\end{remark}

\subsection{Operators on a fixed history tree (optional)}
\label{subsec:tree-operators}

In addition to tier-wise state spaces (fields on $\mathcal D_n$), one may also
consider dynamics \emph{within} a fixed realized history by placing operators on
the internal nodes of its tree.

\paragraph{Node Hilbert space.}
Fix $w\in\mathcal D_n$ and let $T(w)$ be its associated full binary tree. Write
$V_w:=\mathrm{Int}(T(w))$ and consider $\ell^2(V_w)$ with inner product
$\langle \psi,\phi\rangle := \sum_{v\in V_w}\overline{\psi(v)}\,\phi(v)$.

\paragraph{Adjacency and Laplacian.}
Let $G_w=(V_w,E_w)$ be any finite undirected graph on $V_w$ (for example, connect
each internal node to its internal children). Define $A_{G_w}$, $D_{G_w}$, and
the graph Laplacian and generator by
\[
\Delta_{G_w}:=D_{G_w}-A_{G_w},
\qquad
L_{G_w}:=-\Delta_{G_w}.
\]

\paragraph{Heat and Schr\"odinger evolutions.}
The corresponding ``internal-time'' heat equation is
\[
\partial_\tau u = L_{G_w}u,
\]
and the corresponding unitary Schr\"odinger evolution is
\[
i\,\partial_t \psi = -L_{G_w}\psi = \Delta_{G_w}\psi.
\]

\begin{remark}
This within-history operator framework is independent of the tier-growth Markov
dynamics and of coherent summation over histories: it simply records that, once
a graph structure is specified on the internal nodes of a fixed Catalan tree,
standard graph-Laplacian constructions yield discrete diffusion and
Schr\"odinger-type evolutions on that fixed combinatorial background.
\end{remark}

\subsection{Operators on tier slices (optional)}
\label{subsec:tier-operators}

The main text emphasizes two dynamics on the Catalan substrate: tier growth
(prefix extension) and coherent summation over histories. Independently, one may
also consider \emph{slice dynamics} on a fixed tier by endowing the finite set
$\mathcal D_n$ with an auxiliary adjacency graph. This subsection records the
standard operator framework for such constructions.

\paragraph{Tier Hilbert space.}
We take the tier state space to be $\ell^2(\mathcal D_n)$ as in
Section~\ref{subsec:fields}.

\paragraph{Adjacency graphs.}
Let $G_n=(\mathcal D_n,E_n)$ be any finite undirected graph on $\mathcal D_n$.
The choice of $G_n$ is additional structure: different graphs induce different
notions of locality on the tier. A canonical example is the rotation graph (the
associahedron adjacency) on full binary trees, where edges correspond to single
associativity rotations \cite{stanley-catalan,cheneviere2022linear}.

\paragraph{Associahedra and planar tree amplitudes (scattering-amplitude tie-in).}
The associahedron adjacency on $\mathcal D_n$ is also natural from the
perspective of scattering amplitudes. For the planar tree-level sector of
bi-adjoint cubic scalar theory ($\phi^3$), Arkani-Hamed, Bai, He, and Yan
identify an associahedron in planar kinematic space and show that the tree
amplitude is the corresponding canonical form of this positive geometry
\cite{abhy2018scatteringforms}. From this viewpoint, the Catalan enumeration of
planar cubic tree diagrams is not merely counting: the associahedron organizes
factorization channels geometrically, and different triangulations correspond to
different diagrammatic expansions of the same canonical form (see, e.g., the
review \cite{herrmann2022positivegeometry}). For quartic interactions, an
analogous positive-geometry description involves Stokes polytopes rather than
associahedra \cite{banerjee2018stokes}.

\paragraph{Tamari/Dyck/alt-Tamari choices on the same tier.}
The point of introducing an auxiliary graph $G_n$ is that the underlying state
set $\mathcal D_n$ supports multiple natural notions of tier-locality coming
from classical Catalan posets. The rotation graph is the undirected adjacency
underlying the Tamari order; one may likewise equip $\mathcal D_n$ with
adjacency induced by the Dyck (``Stanley'') lattice on Dyck paths, or more
generally by the family of $\delta$-Tamari (alt-Tamari) posets interpolating
between these extremes. These alternatives use different covering relations on
the same Catalan tier and therefore induce different graph Laplacians
$\Delta_{G_n}$ and different ``free'' tier Hamiltonians, but they live on a
common configuration space $\mathcal D_n$ \cite{stanley-catalan,cheneviere2022linear}.

\paragraph{Linear intervals as ``1D corridors'' in Catalan posets.}
A useful robustness fact is that certain one-dimensional substructures are
invariant across these Catalan posets: Chenevi\`ere proves that, for each fixed
$n$ and each height parameter $k$ (in the sense of \cite{cheneviere2022linear}),
the Tamari lattice and the Dyck lattice have the same number of \emph{linear
intervals} (intervals whose Hasse diagram is a chain), and moreover all
alt-Tamari posets share this same count at each height
\cite{cheneviere2022linear}. In the present language, this says that the number
of ``diamond-free corridors'' (regions with a unique maximal chain) is stable
under a wide class of tier-local adjacency choices, reinforcing the theme that
many distinct dynamics can be layered on a single Catalan substrate without
changing its most rigid combinatorial invariants.

\paragraph{Adjacency and Laplacian.}
Define the adjacency operator $A_{G_n}$ and the degree operator $D_{G_n}$ on
$\ell^2(\mathcal D_n)$ by
\[
(A_{G_n}\psi)(w) := \sum_{w'\sim w} \psi(w'),
\qquad
(D_{G_n}\psi)(w) := \deg(w)\,\psi(w),
\]
where $w'\sim w$ denotes adjacency in $G_n$. The (combinatorial) graph Laplacian is
\[
\Delta_{G_n} := D_{G_n}-A_{G_n},
\]
and the associated diffusion generator is
\[
L_{G_n} := -\Delta_{G_n}.
\]
Then $\Delta_{G_n}$ is self-adjoint and positive semidefinite, while $L_{G_n}$ is
self-adjoint and negative semidefinite.

\paragraph{Discrete heat and Schr\"odinger equations.}
The heat equation on the tier graph is the linear ODE
\[
\partial_\tau u = L_{G_n}u,
\]
with solution $u(\tau)=e^{\tau L_{G_n}}u(0)$. Because $-\Delta_{G_n}$ is
self-adjoint and nonpositive, $e^{\tau L_{G_n}}$ is a contraction semigroup.
The corresponding unitary ``free'' Schr\"odinger evolution is
\[
i\,\partial_t \psi = -L_{G_n}\psi = \Delta_{G_n}\psi,
\]
with solution $\psi(t)=e^{-it\Delta_{G_n}}\psi(0)$.

\begin{remark}
This optional tier-graph framework does not fix a preferred choice of adjacency
$G_n$ and is not used elsewhere in the paper. Its purpose is to make explicit
that, once a tier-local notion of neighbourhood is specified, discrete diffusion
and Schr\"odinger-type evolutions on the Catalan state space follow by standard
graph-Laplacian constructions (compare Remark~\ref{rem:dyck-conditioned-drift-scaling}
and Remark~\ref{rem:dyck-conditioned-kernel-doob} for the tier-growth Markov
structure induced by Dyck conditioning).
\end{remark}

% ---------------------------------------------------------------------------
% Bibitems used only in this supplement (kept here for copy/paste convenience).
% If you re-`\input{supplemental-operators.tex}` into the main paper, these
% entries should live in the main `thebibliography` environment.
% ---------------------------------------------------------------------------
\iffalse
\begin{thebibliography}{99}

\bibitem{abhy2018scatteringforms}
N.~Arkani-Hamed, Y.~Bai, S.~He, and G.~Yan.
\newblock Scattering forms and the positive geometry of kinematics, color and
the worldsheet.
\newblock {\em JHEP} \textbf{05} (2018) 096.
\newblock arXiv:1711.09102.

\bibitem{banerjee2018stokes}
P.~Banerjee, A.~Laddha, and P.~Raman.
\newblock Stokes polytopes: The positive geometry for $\phi^4$ interactions.
\newblock arXiv:1811.05904, 2018.

\bibitem{herrmann2022positivegeometry}
E.~Herrmann and J.~Trnka.
\newblock Positive geometry of scattering amplitudes.
\newblock arXiv:2203.13018, 2022.

\end{thebibliography}
\fi
` into the main paper, these
% entries should live in the main `thebibliography` environment.
% ---------------------------------------------------------------------------
\iffalse
\begin{thebibliography}{99}

\bibitem{abhy2018scatteringforms}
N.~Arkani-Hamed, Y.~Bai, S.~He, and G.~Yan.
\newblock Scattering forms and the positive geometry of kinematics, color and
the worldsheet.
\newblock {\em JHEP} \textbf{05} (2018) 096.
\newblock arXiv:1711.09102.

\bibitem{banerjee2018stokes}
P.~Banerjee, A.~Laddha, and P.~Raman.
\newblock Stokes polytopes: The positive geometry for $\phi^4$ interactions.
\newblock arXiv:1811.05904, 2018.

\bibitem{herrmann2022positivegeometry}
E.~Herrmann and J.~Trnka.
\newblock Positive geometry of scattering amplitudes.
\newblock arXiv:2203.13018, 2022.

\end{thebibliography}
\fi
` at the former location in
% Appendix "Additional Technical Notes".

\subsection{Fields on words, prefixes, and nodes (optional)}
\label{subsec:fields}

We use the word ``field'' as shorthand for a complex-valued function on one of
the Catalan objects already in play. Several closely related state spaces are
useful in different contexts.

\paragraph{Fields on completed histories (fixed tier).}
Fix $n$ and consider a function $\Phi_n:\mathcal D_n\to\mathbb{C}$ assigning an
amplitude (or observable value) to each completed history $w\in\mathcal D_n$.
The associated Hilbert space is $\ell^2(\mathcal D_n)$ with inner product
\[
\langle \psi,\phi\rangle := \sum_{w\in\mathcal D_n} \overline{\psi(w)}\,\phi(w).
\]

\paragraph{Fields on prefixes (the full cone).}
Let $\mathcal{C}$ denote the set of Dyck prefixes (admissible partial histories).
A prefix field is a function $\Phi:\mathcal{C}\to\mathbb{C}$, which may be
restricted to a fixed length slice
$\mathcal{C}^{(k)}:=\{p\in\mathcal{C}: |p|=k\}$ when needed.

\paragraph{Fields on nodes of a fixed tree.}
Given $w\in\mathcal D_n$, let $T(w)$ be its associated full binary tree. A
node field is a function $\phi_w:\mathrm{Int}(T(w))\to\mathbb{C}$ on the internal
nodes of that tree.

\begin{remark}
These notions live on different objects (tiers, the prefix poset, or a single
tree) and are independent of any within-tier ordering convention on
$\mathcal D_n$.
\end{remark}

\subsection{Subtree indicators as a multiscale spanning family (optional)}
\label{subsec:subtree-indicators}

Let $T$ be a finite rooted tree and write $\mathrm{Int}(T)$ for its internal
nodes. Each $v\in\mathrm{Int}(T)$ determines a rooted subtree $T_v$, and hence a
subset $\mathrm{Int}(T_v)\subseteq \mathrm{Int}(T)$. Define the subtree indicator
\[
\chi_v:\mathrm{Int}(T)\to\{0,1\},
\qquad
\chi_v(u):=\mathbf{1}\{u\in \mathrm{Int}(T_v)\}.
\]

\begin{lemma}[Subtree indicators form a basis]
\label{lem:subtree-indicator-basis}
The family $\{\chi_v: v\in \mathrm{Int}(T)\}$ is a basis of the vector space of
complex-valued functions on $\mathrm{Int}(T)$.
\end{lemma}

\begin{proof}
Order the internal nodes by nonincreasing depth (deepest first), and let $M$ be
the square matrix with entries $M_{uv}:=\chi_v(u)$. Then $M_{vv}=1$ for all $v$,
while $M_{uv}=0$ whenever $u$ precedes $v$ in this order (a node cannot be a
descendant of a deeper node). Thus $M$ is triangular with ones on the diagonal,
hence invertible. Therefore the indicators are linearly independent and, since
their number equals $\#\mathrm{Int}(T)$, they form a basis.
\end{proof}

\begin{corollary}[Explicit inversion]
\label{cor:subtree-indicator-inversion}
Let $f:\mathrm{Int}(T)\to\mathbb{C}$ be any function. There is a unique family
of coefficients $\{a_v\}_{v\in\mathrm{Int}(T)}$ such that
\[
f \;=\; \sum_{v\in\mathrm{Int}(T)} a_v\,\chi_v.
\]
Writing $\mathrm{par}(v)$ for the parent of $v$ (for $v\neq \mathrm{root}(T)$),
these coefficients are given by
\[
a_{\mathrm{root}(T)} = f(\mathrm{root}(T)),
\qquad
a_v = f(v)-f(\mathrm{par}(v)) \quad (v\neq \mathrm{root}(T)).
\]
\end{corollary}

\begin{proof}
For each $u\in\mathrm{Int}(T)$,
$(\sum_v a_v\chi_v)(u)=\sum_{v:\,u\in\mathrm{Int}(T_v)} a_v
=\sum_{v\preceq u} a_v$, where $v\preceq u$ means that $v$ is an ancestor of
$u$. With the stated choice of coefficients, this ancestor sum telescopes along
the unique root-to-$u$ chain to yield $f(u)$. Uniqueness follows from
Lemma~\ref{lem:subtree-indicator-basis}.
\end{proof}

\begin{remark}
This basis is ``multiscale'': indicators of deep subtrees localize to fine
regions of $T$, while indicators near the root encode coarse structure. Any
choice of orthonormalization yields an orthonormal basis adapted to the rooted
tree geometry.
\end{remark}

\subsection{Operators on a fixed history tree (optional)}
\label{subsec:tree-operators}

In addition to tier-wise state spaces (fields on $\mathcal D_n$), one may also
consider dynamics \emph{within} a fixed realized history by placing operators on
the internal nodes of its tree.

\paragraph{Node Hilbert space.}
Fix $w\in\mathcal D_n$ and let $T(w)$ be its associated full binary tree. Write
$V_w:=\mathrm{Int}(T(w))$ and consider $\ell^2(V_w)$ with inner product
$\langle \psi,\phi\rangle := \sum_{v\in V_w}\overline{\psi(v)}\,\phi(v)$.

\paragraph{Adjacency and Laplacian.}
Let $G_w=(V_w,E_w)$ be any finite undirected graph on $V_w$ (for example, connect
each internal node to its internal children). Define $A_{G_w}$, $D_{G_w}$, and
the graph Laplacian and generator by
\[
\Delta_{G_w}:=D_{G_w}-A_{G_w},
\qquad
L_{G_w}:=-\Delta_{G_w}.
\]

\paragraph{Heat and Schr\"odinger evolutions.}
The corresponding ``internal-time'' heat equation is
\[
\partial_\tau u = L_{G_w}u,
\]
and the corresponding unitary Schr\"odinger evolution is
\[
i\,\partial_t \psi = -L_{G_w}\psi = \Delta_{G_w}\psi.
\]

\begin{remark}
This within-history operator framework is independent of the tier-growth Markov
dynamics and of coherent summation over histories: it simply records that, once
a graph structure is specified on the internal nodes of a fixed Catalan tree,
standard graph-Laplacian constructions yield discrete diffusion and
Schr\"odinger-type evolutions on that fixed combinatorial background.
\end{remark}

\subsection{Operators on tier slices (optional)}
\label{subsec:tier-operators}

The main text emphasizes two dynamics on the Catalan substrate: tier growth
(prefix extension) and coherent summation over histories. Independently, one may
also consider \emph{slice dynamics} on a fixed tier by endowing the finite set
$\mathcal D_n$ with an auxiliary adjacency graph. This subsection records the
standard operator framework for such constructions.

\paragraph{Tier Hilbert space.}
We take the tier state space to be $\ell^2(\mathcal D_n)$ as in
Section~\ref{subsec:fields}.

\paragraph{Adjacency graphs.}
Let $G_n=(\mathcal D_n,E_n)$ be any finite undirected graph on $\mathcal D_n$.
The choice of $G_n$ is additional structure: different graphs induce different
notions of locality on the tier. A canonical example is the rotation graph (the
associahedron adjacency) on full binary trees, where edges correspond to single
associativity rotations \cite{stanley-catalan,cheneviere2022linear}.

\paragraph{Associahedra and planar tree amplitudes (scattering-amplitude tie-in).}
The associahedron adjacency on $\mathcal D_n$ is also natural from the
perspective of scattering amplitudes. For the planar tree-level sector of
bi-adjoint cubic scalar theory ($\phi^3$), Arkani-Hamed, Bai, He, and Yan
identify an associahedron in planar kinematic space and show that the tree
amplitude is the corresponding canonical form of this positive geometry
\cite{abhy2018scatteringforms}. From this viewpoint, the Catalan enumeration of
planar cubic tree diagrams is not merely counting: the associahedron organizes
factorization channels geometrically, and different triangulations correspond to
different diagrammatic expansions of the same canonical form (see, e.g., the
review \cite{herrmann2022positivegeometry}). For quartic interactions, an
analogous positive-geometry description involves Stokes polytopes rather than
associahedra \cite{banerjee2018stokes}.

\paragraph{Tamari/Dyck/alt-Tamari choices on the same tier.}
The point of introducing an auxiliary graph $G_n$ is that the underlying state
set $\mathcal D_n$ supports multiple natural notions of tier-locality coming
from classical Catalan posets. The rotation graph is the undirected adjacency
underlying the Tamari order; one may likewise equip $\mathcal D_n$ with
adjacency induced by the Dyck (``Stanley'') lattice on Dyck paths, or more
generally by the family of $\delta$-Tamari (alt-Tamari) posets interpolating
between these extremes. These alternatives use different covering relations on
the same Catalan tier and therefore induce different graph Laplacians
$\Delta_{G_n}$ and different ``free'' tier Hamiltonians, but they live on a
common configuration space $\mathcal D_n$ \cite{stanley-catalan,cheneviere2022linear}.

\paragraph{Linear intervals as ``1D corridors'' in Catalan posets.}
A useful robustness fact is that certain one-dimensional substructures are
invariant across these Catalan posets: Chenevi\`ere proves that, for each fixed
$n$ and each height parameter $k$ (in the sense of \cite{cheneviere2022linear}),
the Tamari lattice and the Dyck lattice have the same number of \emph{linear
intervals} (intervals whose Hasse diagram is a chain), and moreover all
alt-Tamari posets share this same count at each height
\cite{cheneviere2022linear}. In the present language, this says that the number
of ``diamond-free corridors'' (regions with a unique maximal chain) is stable
under a wide class of tier-local adjacency choices, reinforcing the theme that
many distinct dynamics can be layered on a single Catalan substrate without
changing its most rigid combinatorial invariants.

\paragraph{Adjacency and Laplacian.}
Define the adjacency operator $A_{G_n}$ and the degree operator $D_{G_n}$ on
$\ell^2(\mathcal D_n)$ by
\[
(A_{G_n}\psi)(w) := \sum_{w'\sim w} \psi(w'),
\qquad
(D_{G_n}\psi)(w) := \deg(w)\,\psi(w),
\]
where $w'\sim w$ denotes adjacency in $G_n$. The (combinatorial) graph Laplacian is
\[
\Delta_{G_n} := D_{G_n}-A_{G_n},
\]
and the associated diffusion generator is
\[
L_{G_n} := -\Delta_{G_n}.
\]
Then $\Delta_{G_n}$ is self-adjoint and positive semidefinite, while $L_{G_n}$ is
self-adjoint and negative semidefinite.

\paragraph{Discrete heat and Schr\"odinger equations.}
The heat equation on the tier graph is the linear ODE
\[
\partial_\tau u = L_{G_n}u,
\]
with solution $u(\tau)=e^{\tau L_{G_n}}u(0)$. Because $-\Delta_{G_n}$ is
self-adjoint and nonpositive, $e^{\tau L_{G_n}}$ is a contraction semigroup.
The corresponding unitary ``free'' Schr\"odinger evolution is
\[
i\,\partial_t \psi = -L_{G_n}\psi = \Delta_{G_n}\psi,
\]
with solution $\psi(t)=e^{-it\Delta_{G_n}}\psi(0)$.

\begin{remark}
This optional tier-graph framework does not fix a preferred choice of adjacency
$G_n$ and is not used elsewhere in the paper. Its purpose is to make explicit
that, once a tier-local notion of neighbourhood is specified, discrete diffusion
and Schr\"odinger-type evolutions on the Catalan state space follow by standard
graph-Laplacian constructions (compare Remark~\ref{rem:dyck-conditioned-drift-scaling}
and Remark~\ref{rem:dyck-conditioned-kernel-doob} for the tier-growth Markov
structure induced by Dyck conditioning).
\end{remark}

% ---------------------------------------------------------------------------
% Bibitems used only in this supplement (kept here for copy/paste convenience).
% If you re-`% Supplemental material extracted from `docs/catalan-light-cone.tex`.
% This block was removed from the compiled arXiv PDF to keep the main v1 lean.
% Intended use: `\% Supplemental material extracted from `docs/catalan-light-cone.tex`.
% This block was removed from the compiled arXiv PDF to keep the main v1 lean.
% Intended use: `\\input{supplemental-operators.tex}` at the former location in
% Appendix "Additional Technical Notes".

\subsection{Fields on words, prefixes, and nodes (optional)}
\label{subsec:fields}

We use the word ``field'' as shorthand for a complex-valued function on one of
the Catalan objects already in play. Several closely related state spaces are
useful in different contexts.

\paragraph{Fields on completed histories (fixed tier).}
Fix $n$ and consider a function $\Phi_n:\mathcal D_n\to\mathbb{C}$ assigning an
amplitude (or observable value) to each completed history $w\in\mathcal D_n$.
The associated Hilbert space is $\ell^2(\mathcal D_n)$ with inner product
\[
\langle \psi,\phi\rangle := \sum_{w\in\mathcal D_n} \overline{\psi(w)}\,\phi(w).
\]

\paragraph{Fields on prefixes (the full cone).}
Let $\mathcal{C}$ denote the set of Dyck prefixes (admissible partial histories).
A prefix field is a function $\Phi:\mathcal{C}\to\mathbb{C}$, which may be
restricted to a fixed length slice
$\mathcal{C}^{(k)}:=\{p\in\mathcal{C}: |p|=k\}$ when needed.

\paragraph{Fields on nodes of a fixed tree.}
Given $w\in\mathcal D_n$, let $T(w)$ be its associated full binary tree. A
node field is a function $\phi_w:\mathrm{Int}(T(w))\to\mathbb{C}$ on the internal
nodes of that tree.

\begin{remark}
These notions live on different objects (tiers, the prefix poset, or a single
tree) and are independent of any within-tier ordering convention on
$\mathcal D_n$.
\end{remark}

\subsection{Subtree indicators as a multiscale spanning family (optional)}
\label{subsec:subtree-indicators}

Let $T$ be a finite rooted tree and write $\mathrm{Int}(T)$ for its internal
nodes. Each $v\in\mathrm{Int}(T)$ determines a rooted subtree $T_v$, and hence a
subset $\mathrm{Int}(T_v)\subseteq \mathrm{Int}(T)$. Define the subtree indicator
\[
\chi_v:\mathrm{Int}(T)\to\{0,1\},
\qquad
\chi_v(u):=\mathbf{1}\{u\in \mathrm{Int}(T_v)\}.
\]

\begin{lemma}[Subtree indicators form a basis]
\label{lem:subtree-indicator-basis}
The family $\{\chi_v: v\in \mathrm{Int}(T)\}$ is a basis of the vector space of
complex-valued functions on $\mathrm{Int}(T)$.
\end{lemma}

\begin{proof}
Order the internal nodes by nonincreasing depth (deepest first), and let $M$ be
the square matrix with entries $M_{uv}:=\chi_v(u)$. Then $M_{vv}=1$ for all $v$,
while $M_{uv}=0$ whenever $u$ precedes $v$ in this order (a node cannot be a
descendant of a deeper node). Thus $M$ is triangular with ones on the diagonal,
hence invertible. Therefore the indicators are linearly independent and, since
their number equals $\#\mathrm{Int}(T)$, they form a basis.
\end{proof}

\begin{corollary}[Explicit inversion]
\label{cor:subtree-indicator-inversion}
Let $f:\mathrm{Int}(T)\to\mathbb{C}$ be any function. There is a unique family
of coefficients $\{a_v\}_{v\in\mathrm{Int}(T)}$ such that
\[
f \;=\; \sum_{v\in\mathrm{Int}(T)} a_v\,\chi_v.
\]
Writing $\mathrm{par}(v)$ for the parent of $v$ (for $v\neq \mathrm{root}(T)$),
these coefficients are given by
\[
a_{\mathrm{root}(T)} = f(\mathrm{root}(T)),
\qquad
a_v = f(v)-f(\mathrm{par}(v)) \quad (v\neq \mathrm{root}(T)).
\]
\end{corollary}

\begin{proof}
For each $u\in\mathrm{Int}(T)$,
$(\sum_v a_v\chi_v)(u)=\sum_{v:\,u\in\mathrm{Int}(T_v)} a_v
=\sum_{v\preceq u} a_v$, where $v\preceq u$ means that $v$ is an ancestor of
$u$. With the stated choice of coefficients, this ancestor sum telescopes along
the unique root-to-$u$ chain to yield $f(u)$. Uniqueness follows from
Lemma~\ref{lem:subtree-indicator-basis}.
\end{proof}

\begin{remark}
This basis is ``multiscale'': indicators of deep subtrees localize to fine
regions of $T$, while indicators near the root encode coarse structure. Any
choice of orthonormalization yields an orthonormal basis adapted to the rooted
tree geometry.
\end{remark}

\subsection{Operators on a fixed history tree (optional)}
\label{subsec:tree-operators}

In addition to tier-wise state spaces (fields on $\mathcal D_n$), one may also
consider dynamics \emph{within} a fixed realized history by placing operators on
the internal nodes of its tree.

\paragraph{Node Hilbert space.}
Fix $w\in\mathcal D_n$ and let $T(w)$ be its associated full binary tree. Write
$V_w:=\mathrm{Int}(T(w))$ and consider $\ell^2(V_w)$ with inner product
$\langle \psi,\phi\rangle := \sum_{v\in V_w}\overline{\psi(v)}\,\phi(v)$.

\paragraph{Adjacency and Laplacian.}
Let $G_w=(V_w,E_w)$ be any finite undirected graph on $V_w$ (for example, connect
each internal node to its internal children). Define $A_{G_w}$, $D_{G_w}$, and
the graph Laplacian and generator by
\[
\Delta_{G_w}:=D_{G_w}-A_{G_w},
\qquad
L_{G_w}:=-\Delta_{G_w}.
\]

\paragraph{Heat and Schr\"odinger evolutions.}
The corresponding ``internal-time'' heat equation is
\[
\partial_\tau u = L_{G_w}u,
\]
and the corresponding unitary Schr\"odinger evolution is
\[
i\,\partial_t \psi = -L_{G_w}\psi = \Delta_{G_w}\psi.
\]

\begin{remark}
This within-history operator framework is independent of the tier-growth Markov
dynamics and of coherent summation over histories: it simply records that, once
a graph structure is specified on the internal nodes of a fixed Catalan tree,
standard graph-Laplacian constructions yield discrete diffusion and
Schr\"odinger-type evolutions on that fixed combinatorial background.
\end{remark}

\subsection{Operators on tier slices (optional)}
\label{subsec:tier-operators}

The main text emphasizes two dynamics on the Catalan substrate: tier growth
(prefix extension) and coherent summation over histories. Independently, one may
also consider \emph{slice dynamics} on a fixed tier by endowing the finite set
$\mathcal D_n$ with an auxiliary adjacency graph. This subsection records the
standard operator framework for such constructions.

\paragraph{Tier Hilbert space.}
We take the tier state space to be $\ell^2(\mathcal D_n)$ as in
Section~\ref{subsec:fields}.

\paragraph{Adjacency graphs.}
Let $G_n=(\mathcal D_n,E_n)$ be any finite undirected graph on $\mathcal D_n$.
The choice of $G_n$ is additional structure: different graphs induce different
notions of locality on the tier. A canonical example is the rotation graph (the
associahedron adjacency) on full binary trees, where edges correspond to single
associativity rotations \cite{stanley-catalan,cheneviere2022linear}.

\paragraph{Associahedra and planar tree amplitudes (scattering-amplitude tie-in).}
The associahedron adjacency on $\mathcal D_n$ is also natural from the
perspective of scattering amplitudes. For the planar tree-level sector of
bi-adjoint cubic scalar theory ($\phi^3$), Arkani-Hamed, Bai, He, and Yan
identify an associahedron in planar kinematic space and show that the tree
amplitude is the corresponding canonical form of this positive geometry
\cite{abhy2018scatteringforms}. From this viewpoint, the Catalan enumeration of
planar cubic tree diagrams is not merely counting: the associahedron organizes
factorization channels geometrically, and different triangulations correspond to
different diagrammatic expansions of the same canonical form (see, e.g., the
review \cite{herrmann2022positivegeometry}). For quartic interactions, an
analogous positive-geometry description involves Stokes polytopes rather than
associahedra \cite{banerjee2018stokes}.

\paragraph{Tamari/Dyck/alt-Tamari choices on the same tier.}
The point of introducing an auxiliary graph $G_n$ is that the underlying state
set $\mathcal D_n$ supports multiple natural notions of tier-locality coming
from classical Catalan posets. The rotation graph is the undirected adjacency
underlying the Tamari order; one may likewise equip $\mathcal D_n$ with
adjacency induced by the Dyck (``Stanley'') lattice on Dyck paths, or more
generally by the family of $\delta$-Tamari (alt-Tamari) posets interpolating
between these extremes. These alternatives use different covering relations on
the same Catalan tier and therefore induce different graph Laplacians
$\Delta_{G_n}$ and different ``free'' tier Hamiltonians, but they live on a
common configuration space $\mathcal D_n$ \cite{stanley-catalan,cheneviere2022linear}.

\paragraph{Linear intervals as ``1D corridors'' in Catalan posets.}
A useful robustness fact is that certain one-dimensional substructures are
invariant across these Catalan posets: Chenevi\`ere proves that, for each fixed
$n$ and each height parameter $k$ (in the sense of \cite{cheneviere2022linear}),
the Tamari lattice and the Dyck lattice have the same number of \emph{linear
intervals} (intervals whose Hasse diagram is a chain), and moreover all
alt-Tamari posets share this same count at each height
\cite{cheneviere2022linear}. In the present language, this says that the number
of ``diamond-free corridors'' (regions with a unique maximal chain) is stable
under a wide class of tier-local adjacency choices, reinforcing the theme that
many distinct dynamics can be layered on a single Catalan substrate without
changing its most rigid combinatorial invariants.

\paragraph{Adjacency and Laplacian.}
Define the adjacency operator $A_{G_n}$ and the degree operator $D_{G_n}$ on
$\ell^2(\mathcal D_n)$ by
\[
(A_{G_n}\psi)(w) := \sum_{w'\sim w} \psi(w'),
\qquad
(D_{G_n}\psi)(w) := \deg(w)\,\psi(w),
\]
where $w'\sim w$ denotes adjacency in $G_n$. The (combinatorial) graph Laplacian is
\[
\Delta_{G_n} := D_{G_n}-A_{G_n},
\]
and the associated diffusion generator is
\[
L_{G_n} := -\Delta_{G_n}.
\]
Then $\Delta_{G_n}$ is self-adjoint and positive semidefinite, while $L_{G_n}$ is
self-adjoint and negative semidefinite.

\paragraph{Discrete heat and Schr\"odinger equations.}
The heat equation on the tier graph is the linear ODE
\[
\partial_\tau u = L_{G_n}u,
\]
with solution $u(\tau)=e^{\tau L_{G_n}}u(0)$. Because $-\Delta_{G_n}$ is
self-adjoint and nonpositive, $e^{\tau L_{G_n}}$ is a contraction semigroup.
The corresponding unitary ``free'' Schr\"odinger evolution is
\[
i\,\partial_t \psi = -L_{G_n}\psi = \Delta_{G_n}\psi,
\]
with solution $\psi(t)=e^{-it\Delta_{G_n}}\psi(0)$.

\begin{remark}
This optional tier-graph framework does not fix a preferred choice of adjacency
$G_n$ and is not used elsewhere in the paper. Its purpose is to make explicit
that, once a tier-local notion of neighbourhood is specified, discrete diffusion
and Schr\"odinger-type evolutions on the Catalan state space follow by standard
graph-Laplacian constructions (compare Remark~\ref{rem:dyck-conditioned-drift-scaling}
and Remark~\ref{rem:dyck-conditioned-kernel-doob} for the tier-growth Markov
structure induced by Dyck conditioning).
\end{remark}

% ---------------------------------------------------------------------------
% Bibitems used only in this supplement (kept here for copy/paste convenience).
% If you re-`\input{supplemental-operators.tex}` into the main paper, these
% entries should live in the main `thebibliography` environment.
% ---------------------------------------------------------------------------
\iffalse
\begin{thebibliography}{99}

\bibitem{abhy2018scatteringforms}
N.~Arkani-Hamed, Y.~Bai, S.~He, and G.~Yan.
\newblock Scattering forms and the positive geometry of kinematics, color and
the worldsheet.
\newblock {\em JHEP} \textbf{05} (2018) 096.
\newblock arXiv:1711.09102.

\bibitem{banerjee2018stokes}
P.~Banerjee, A.~Laddha, and P.~Raman.
\newblock Stokes polytopes: The positive geometry for $\phi^4$ interactions.
\newblock arXiv:1811.05904, 2018.

\bibitem{herrmann2022positivegeometry}
E.~Herrmann and J.~Trnka.
\newblock Positive geometry of scattering amplitudes.
\newblock arXiv:2203.13018, 2022.

\end{thebibliography}
\fi
` at the former location in
% Appendix "Additional Technical Notes".

\subsection{Fields on words, prefixes, and nodes (optional)}
\label{subsec:fields}

We use the word ``field'' as shorthand for a complex-valued function on one of
the Catalan objects already in play. Several closely related state spaces are
useful in different contexts.

\paragraph{Fields on completed histories (fixed tier).}
Fix $n$ and consider a function $\Phi_n:\mathcal D_n\to\mathbb{C}$ assigning an
amplitude (or observable value) to each completed history $w\in\mathcal D_n$.
The associated Hilbert space is $\ell^2(\mathcal D_n)$ with inner product
\[
\langle \psi,\phi\rangle := \sum_{w\in\mathcal D_n} \overline{\psi(w)}\,\phi(w).
\]

\paragraph{Fields on prefixes (the full cone).}
Let $\mathcal{C}$ denote the set of Dyck prefixes (admissible partial histories).
A prefix field is a function $\Phi:\mathcal{C}\to\mathbb{C}$, which may be
restricted to a fixed length slice
$\mathcal{C}^{(k)}:=\{p\in\mathcal{C}: |p|=k\}$ when needed.

\paragraph{Fields on nodes of a fixed tree.}
Given $w\in\mathcal D_n$, let $T(w)$ be its associated full binary tree. A
node field is a function $\phi_w:\mathrm{Int}(T(w))\to\mathbb{C}$ on the internal
nodes of that tree.

\begin{remark}
These notions live on different objects (tiers, the prefix poset, or a single
tree) and are independent of any within-tier ordering convention on
$\mathcal D_n$.
\end{remark}

\subsection{Subtree indicators as a multiscale spanning family (optional)}
\label{subsec:subtree-indicators}

Let $T$ be a finite rooted tree and write $\mathrm{Int}(T)$ for its internal
nodes. Each $v\in\mathrm{Int}(T)$ determines a rooted subtree $T_v$, and hence a
subset $\mathrm{Int}(T_v)\subseteq \mathrm{Int}(T)$. Define the subtree indicator
\[
\chi_v:\mathrm{Int}(T)\to\{0,1\},
\qquad
\chi_v(u):=\mathbf{1}\{u\in \mathrm{Int}(T_v)\}.
\]

\begin{lemma}[Subtree indicators form a basis]
\label{lem:subtree-indicator-basis}
The family $\{\chi_v: v\in \mathrm{Int}(T)\}$ is a basis of the vector space of
complex-valued functions on $\mathrm{Int}(T)$.
\end{lemma}

\begin{proof}
Order the internal nodes by nonincreasing depth (deepest first), and let $M$ be
the square matrix with entries $M_{uv}:=\chi_v(u)$. Then $M_{vv}=1$ for all $v$,
while $M_{uv}=0$ whenever $u$ precedes $v$ in this order (a node cannot be a
descendant of a deeper node). Thus $M$ is triangular with ones on the diagonal,
hence invertible. Therefore the indicators are linearly independent and, since
their number equals $\#\mathrm{Int}(T)$, they form a basis.
\end{proof}

\begin{corollary}[Explicit inversion]
\label{cor:subtree-indicator-inversion}
Let $f:\mathrm{Int}(T)\to\mathbb{C}$ be any function. There is a unique family
of coefficients $\{a_v\}_{v\in\mathrm{Int}(T)}$ such that
\[
f \;=\; \sum_{v\in\mathrm{Int}(T)} a_v\,\chi_v.
\]
Writing $\mathrm{par}(v)$ for the parent of $v$ (for $v\neq \mathrm{root}(T)$),
these coefficients are given by
\[
a_{\mathrm{root}(T)} = f(\mathrm{root}(T)),
\qquad
a_v = f(v)-f(\mathrm{par}(v)) \quad (v\neq \mathrm{root}(T)).
\]
\end{corollary}

\begin{proof}
For each $u\in\mathrm{Int}(T)$,
$(\sum_v a_v\chi_v)(u)=\sum_{v:\,u\in\mathrm{Int}(T_v)} a_v
=\sum_{v\preceq u} a_v$, where $v\preceq u$ means that $v$ is an ancestor of
$u$. With the stated choice of coefficients, this ancestor sum telescopes along
the unique root-to-$u$ chain to yield $f(u)$. Uniqueness follows from
Lemma~\ref{lem:subtree-indicator-basis}.
\end{proof}

\begin{remark}
This basis is ``multiscale'': indicators of deep subtrees localize to fine
regions of $T$, while indicators near the root encode coarse structure. Any
choice of orthonormalization yields an orthonormal basis adapted to the rooted
tree geometry.
\end{remark}

\subsection{Operators on a fixed history tree (optional)}
\label{subsec:tree-operators}

In addition to tier-wise state spaces (fields on $\mathcal D_n$), one may also
consider dynamics \emph{within} a fixed realized history by placing operators on
the internal nodes of its tree.

\paragraph{Node Hilbert space.}
Fix $w\in\mathcal D_n$ and let $T(w)$ be its associated full binary tree. Write
$V_w:=\mathrm{Int}(T(w))$ and consider $\ell^2(V_w)$ with inner product
$\langle \psi,\phi\rangle := \sum_{v\in V_w}\overline{\psi(v)}\,\phi(v)$.

\paragraph{Adjacency and Laplacian.}
Let $G_w=(V_w,E_w)$ be any finite undirected graph on $V_w$ (for example, connect
each internal node to its internal children). Define $A_{G_w}$, $D_{G_w}$, and
the graph Laplacian and generator by
\[
\Delta_{G_w}:=D_{G_w}-A_{G_w},
\qquad
L_{G_w}:=-\Delta_{G_w}.
\]

\paragraph{Heat and Schr\"odinger evolutions.}
The corresponding ``internal-time'' heat equation is
\[
\partial_\tau u = L_{G_w}u,
\]
and the corresponding unitary Schr\"odinger evolution is
\[
i\,\partial_t \psi = -L_{G_w}\psi = \Delta_{G_w}\psi.
\]

\begin{remark}
This within-history operator framework is independent of the tier-growth Markov
dynamics and of coherent summation over histories: it simply records that, once
a graph structure is specified on the internal nodes of a fixed Catalan tree,
standard graph-Laplacian constructions yield discrete diffusion and
Schr\"odinger-type evolutions on that fixed combinatorial background.
\end{remark}

\subsection{Operators on tier slices (optional)}
\label{subsec:tier-operators}

The main text emphasizes two dynamics on the Catalan substrate: tier growth
(prefix extension) and coherent summation over histories. Independently, one may
also consider \emph{slice dynamics} on a fixed tier by endowing the finite set
$\mathcal D_n$ with an auxiliary adjacency graph. This subsection records the
standard operator framework for such constructions.

\paragraph{Tier Hilbert space.}
We take the tier state space to be $\ell^2(\mathcal D_n)$ as in
Section~\ref{subsec:fields}.

\paragraph{Adjacency graphs.}
Let $G_n=(\mathcal D_n,E_n)$ be any finite undirected graph on $\mathcal D_n$.
The choice of $G_n$ is additional structure: different graphs induce different
notions of locality on the tier. A canonical example is the rotation graph (the
associahedron adjacency) on full binary trees, where edges correspond to single
associativity rotations \cite{stanley-catalan,cheneviere2022linear}.

\paragraph{Associahedra and planar tree amplitudes (scattering-amplitude tie-in).}
The associahedron adjacency on $\mathcal D_n$ is also natural from the
perspective of scattering amplitudes. For the planar tree-level sector of
bi-adjoint cubic scalar theory ($\phi^3$), Arkani-Hamed, Bai, He, and Yan
identify an associahedron in planar kinematic space and show that the tree
amplitude is the corresponding canonical form of this positive geometry
\cite{abhy2018scatteringforms}. From this viewpoint, the Catalan enumeration of
planar cubic tree diagrams is not merely counting: the associahedron organizes
factorization channels geometrically, and different triangulations correspond to
different diagrammatic expansions of the same canonical form (see, e.g., the
review \cite{herrmann2022positivegeometry}). For quartic interactions, an
analogous positive-geometry description involves Stokes polytopes rather than
associahedra \cite{banerjee2018stokes}.

\paragraph{Tamari/Dyck/alt-Tamari choices on the same tier.}
The point of introducing an auxiliary graph $G_n$ is that the underlying state
set $\mathcal D_n$ supports multiple natural notions of tier-locality coming
from classical Catalan posets. The rotation graph is the undirected adjacency
underlying the Tamari order; one may likewise equip $\mathcal D_n$ with
adjacency induced by the Dyck (``Stanley'') lattice on Dyck paths, or more
generally by the family of $\delta$-Tamari (alt-Tamari) posets interpolating
between these extremes. These alternatives use different covering relations on
the same Catalan tier and therefore induce different graph Laplacians
$\Delta_{G_n}$ and different ``free'' tier Hamiltonians, but they live on a
common configuration space $\mathcal D_n$ \cite{stanley-catalan,cheneviere2022linear}.

\paragraph{Linear intervals as ``1D corridors'' in Catalan posets.}
A useful robustness fact is that certain one-dimensional substructures are
invariant across these Catalan posets: Chenevi\`ere proves that, for each fixed
$n$ and each height parameter $k$ (in the sense of \cite{cheneviere2022linear}),
the Tamari lattice and the Dyck lattice have the same number of \emph{linear
intervals} (intervals whose Hasse diagram is a chain), and moreover all
alt-Tamari posets share this same count at each height
\cite{cheneviere2022linear}. In the present language, this says that the number
of ``diamond-free corridors'' (regions with a unique maximal chain) is stable
under a wide class of tier-local adjacency choices, reinforcing the theme that
many distinct dynamics can be layered on a single Catalan substrate without
changing its most rigid combinatorial invariants.

\paragraph{Adjacency and Laplacian.}
Define the adjacency operator $A_{G_n}$ and the degree operator $D_{G_n}$ on
$\ell^2(\mathcal D_n)$ by
\[
(A_{G_n}\psi)(w) := \sum_{w'\sim w} \psi(w'),
\qquad
(D_{G_n}\psi)(w) := \deg(w)\,\psi(w),
\]
where $w'\sim w$ denotes adjacency in $G_n$. The (combinatorial) graph Laplacian is
\[
\Delta_{G_n} := D_{G_n}-A_{G_n},
\]
and the associated diffusion generator is
\[
L_{G_n} := -\Delta_{G_n}.
\]
Then $\Delta_{G_n}$ is self-adjoint and positive semidefinite, while $L_{G_n}$ is
self-adjoint and negative semidefinite.

\paragraph{Discrete heat and Schr\"odinger equations.}
The heat equation on the tier graph is the linear ODE
\[
\partial_\tau u = L_{G_n}u,
\]
with solution $u(\tau)=e^{\tau L_{G_n}}u(0)$. Because $-\Delta_{G_n}$ is
self-adjoint and nonpositive, $e^{\tau L_{G_n}}$ is a contraction semigroup.
The corresponding unitary ``free'' Schr\"odinger evolution is
\[
i\,\partial_t \psi = -L_{G_n}\psi = \Delta_{G_n}\psi,
\]
with solution $\psi(t)=e^{-it\Delta_{G_n}}\psi(0)$.

\begin{remark}
This optional tier-graph framework does not fix a preferred choice of adjacency
$G_n$ and is not used elsewhere in the paper. Its purpose is to make explicit
that, once a tier-local notion of neighbourhood is specified, discrete diffusion
and Schr\"odinger-type evolutions on the Catalan state space follow by standard
graph-Laplacian constructions (compare Remark~\ref{rem:dyck-conditioned-drift-scaling}
and Remark~\ref{rem:dyck-conditioned-kernel-doob} for the tier-growth Markov
structure induced by Dyck conditioning).
\end{remark}

% ---------------------------------------------------------------------------
% Bibitems used only in this supplement (kept here for copy/paste convenience).
% If you re-`% Supplemental material extracted from `docs/catalan-light-cone.tex`.
% This block was removed from the compiled arXiv PDF to keep the main v1 lean.
% Intended use: `\\input{supplemental-operators.tex}` at the former location in
% Appendix "Additional Technical Notes".

\subsection{Fields on words, prefixes, and nodes (optional)}
\label{subsec:fields}

We use the word ``field'' as shorthand for a complex-valued function on one of
the Catalan objects already in play. Several closely related state spaces are
useful in different contexts.

\paragraph{Fields on completed histories (fixed tier).}
Fix $n$ and consider a function $\Phi_n:\mathcal D_n\to\mathbb{C}$ assigning an
amplitude (or observable value) to each completed history $w\in\mathcal D_n$.
The associated Hilbert space is $\ell^2(\mathcal D_n)$ with inner product
\[
\langle \psi,\phi\rangle := \sum_{w\in\mathcal D_n} \overline{\psi(w)}\,\phi(w).
\]

\paragraph{Fields on prefixes (the full cone).}
Let $\mathcal{C}$ denote the set of Dyck prefixes (admissible partial histories).
A prefix field is a function $\Phi:\mathcal{C}\to\mathbb{C}$, which may be
restricted to a fixed length slice
$\mathcal{C}^{(k)}:=\{p\in\mathcal{C}: |p|=k\}$ when needed.

\paragraph{Fields on nodes of a fixed tree.}
Given $w\in\mathcal D_n$, let $T(w)$ be its associated full binary tree. A
node field is a function $\phi_w:\mathrm{Int}(T(w))\to\mathbb{C}$ on the internal
nodes of that tree.

\begin{remark}
These notions live on different objects (tiers, the prefix poset, or a single
tree) and are independent of any within-tier ordering convention on
$\mathcal D_n$.
\end{remark}

\subsection{Subtree indicators as a multiscale spanning family (optional)}
\label{subsec:subtree-indicators}

Let $T$ be a finite rooted tree and write $\mathrm{Int}(T)$ for its internal
nodes. Each $v\in\mathrm{Int}(T)$ determines a rooted subtree $T_v$, and hence a
subset $\mathrm{Int}(T_v)\subseteq \mathrm{Int}(T)$. Define the subtree indicator
\[
\chi_v:\mathrm{Int}(T)\to\{0,1\},
\qquad
\chi_v(u):=\mathbf{1}\{u\in \mathrm{Int}(T_v)\}.
\]

\begin{lemma}[Subtree indicators form a basis]
\label{lem:subtree-indicator-basis}
The family $\{\chi_v: v\in \mathrm{Int}(T)\}$ is a basis of the vector space of
complex-valued functions on $\mathrm{Int}(T)$.
\end{lemma}

\begin{proof}
Order the internal nodes by nonincreasing depth (deepest first), and let $M$ be
the square matrix with entries $M_{uv}:=\chi_v(u)$. Then $M_{vv}=1$ for all $v$,
while $M_{uv}=0$ whenever $u$ precedes $v$ in this order (a node cannot be a
descendant of a deeper node). Thus $M$ is triangular with ones on the diagonal,
hence invertible. Therefore the indicators are linearly independent and, since
their number equals $\#\mathrm{Int}(T)$, they form a basis.
\end{proof}

\begin{corollary}[Explicit inversion]
\label{cor:subtree-indicator-inversion}
Let $f:\mathrm{Int}(T)\to\mathbb{C}$ be any function. There is a unique family
of coefficients $\{a_v\}_{v\in\mathrm{Int}(T)}$ such that
\[
f \;=\; \sum_{v\in\mathrm{Int}(T)} a_v\,\chi_v.
\]
Writing $\mathrm{par}(v)$ for the parent of $v$ (for $v\neq \mathrm{root}(T)$),
these coefficients are given by
\[
a_{\mathrm{root}(T)} = f(\mathrm{root}(T)),
\qquad
a_v = f(v)-f(\mathrm{par}(v)) \quad (v\neq \mathrm{root}(T)).
\]
\end{corollary}

\begin{proof}
For each $u\in\mathrm{Int}(T)$,
$(\sum_v a_v\chi_v)(u)=\sum_{v:\,u\in\mathrm{Int}(T_v)} a_v
=\sum_{v\preceq u} a_v$, where $v\preceq u$ means that $v$ is an ancestor of
$u$. With the stated choice of coefficients, this ancestor sum telescopes along
the unique root-to-$u$ chain to yield $f(u)$. Uniqueness follows from
Lemma~\ref{lem:subtree-indicator-basis}.
\end{proof}

\begin{remark}
This basis is ``multiscale'': indicators of deep subtrees localize to fine
regions of $T$, while indicators near the root encode coarse structure. Any
choice of orthonormalization yields an orthonormal basis adapted to the rooted
tree geometry.
\end{remark}

\subsection{Operators on a fixed history tree (optional)}
\label{subsec:tree-operators}

In addition to tier-wise state spaces (fields on $\mathcal D_n$), one may also
consider dynamics \emph{within} a fixed realized history by placing operators on
the internal nodes of its tree.

\paragraph{Node Hilbert space.}
Fix $w\in\mathcal D_n$ and let $T(w)$ be its associated full binary tree. Write
$V_w:=\mathrm{Int}(T(w))$ and consider $\ell^2(V_w)$ with inner product
$\langle \psi,\phi\rangle := \sum_{v\in V_w}\overline{\psi(v)}\,\phi(v)$.

\paragraph{Adjacency and Laplacian.}
Let $G_w=(V_w,E_w)$ be any finite undirected graph on $V_w$ (for example, connect
each internal node to its internal children). Define $A_{G_w}$, $D_{G_w}$, and
the graph Laplacian and generator by
\[
\Delta_{G_w}:=D_{G_w}-A_{G_w},
\qquad
L_{G_w}:=-\Delta_{G_w}.
\]

\paragraph{Heat and Schr\"odinger evolutions.}
The corresponding ``internal-time'' heat equation is
\[
\partial_\tau u = L_{G_w}u,
\]
and the corresponding unitary Schr\"odinger evolution is
\[
i\,\partial_t \psi = -L_{G_w}\psi = \Delta_{G_w}\psi.
\]

\begin{remark}
This within-history operator framework is independent of the tier-growth Markov
dynamics and of coherent summation over histories: it simply records that, once
a graph structure is specified on the internal nodes of a fixed Catalan tree,
standard graph-Laplacian constructions yield discrete diffusion and
Schr\"odinger-type evolutions on that fixed combinatorial background.
\end{remark}

\subsection{Operators on tier slices (optional)}
\label{subsec:tier-operators}

The main text emphasizes two dynamics on the Catalan substrate: tier growth
(prefix extension) and coherent summation over histories. Independently, one may
also consider \emph{slice dynamics} on a fixed tier by endowing the finite set
$\mathcal D_n$ with an auxiliary adjacency graph. This subsection records the
standard operator framework for such constructions.

\paragraph{Tier Hilbert space.}
We take the tier state space to be $\ell^2(\mathcal D_n)$ as in
Section~\ref{subsec:fields}.

\paragraph{Adjacency graphs.}
Let $G_n=(\mathcal D_n,E_n)$ be any finite undirected graph on $\mathcal D_n$.
The choice of $G_n$ is additional structure: different graphs induce different
notions of locality on the tier. A canonical example is the rotation graph (the
associahedron adjacency) on full binary trees, where edges correspond to single
associativity rotations \cite{stanley-catalan,cheneviere2022linear}.

\paragraph{Associahedra and planar tree amplitudes (scattering-amplitude tie-in).}
The associahedron adjacency on $\mathcal D_n$ is also natural from the
perspective of scattering amplitudes. For the planar tree-level sector of
bi-adjoint cubic scalar theory ($\phi^3$), Arkani-Hamed, Bai, He, and Yan
identify an associahedron in planar kinematic space and show that the tree
amplitude is the corresponding canonical form of this positive geometry
\cite{abhy2018scatteringforms}. From this viewpoint, the Catalan enumeration of
planar cubic tree diagrams is not merely counting: the associahedron organizes
factorization channels geometrically, and different triangulations correspond to
different diagrammatic expansions of the same canonical form (see, e.g., the
review \cite{herrmann2022positivegeometry}). For quartic interactions, an
analogous positive-geometry description involves Stokes polytopes rather than
associahedra \cite{banerjee2018stokes}.

\paragraph{Tamari/Dyck/alt-Tamari choices on the same tier.}
The point of introducing an auxiliary graph $G_n$ is that the underlying state
set $\mathcal D_n$ supports multiple natural notions of tier-locality coming
from classical Catalan posets. The rotation graph is the undirected adjacency
underlying the Tamari order; one may likewise equip $\mathcal D_n$ with
adjacency induced by the Dyck (``Stanley'') lattice on Dyck paths, or more
generally by the family of $\delta$-Tamari (alt-Tamari) posets interpolating
between these extremes. These alternatives use different covering relations on
the same Catalan tier and therefore induce different graph Laplacians
$\Delta_{G_n}$ and different ``free'' tier Hamiltonians, but they live on a
common configuration space $\mathcal D_n$ \cite{stanley-catalan,cheneviere2022linear}.

\paragraph{Linear intervals as ``1D corridors'' in Catalan posets.}
A useful robustness fact is that certain one-dimensional substructures are
invariant across these Catalan posets: Chenevi\`ere proves that, for each fixed
$n$ and each height parameter $k$ (in the sense of \cite{cheneviere2022linear}),
the Tamari lattice and the Dyck lattice have the same number of \emph{linear
intervals} (intervals whose Hasse diagram is a chain), and moreover all
alt-Tamari posets share this same count at each height
\cite{cheneviere2022linear}. In the present language, this says that the number
of ``diamond-free corridors'' (regions with a unique maximal chain) is stable
under a wide class of tier-local adjacency choices, reinforcing the theme that
many distinct dynamics can be layered on a single Catalan substrate without
changing its most rigid combinatorial invariants.

\paragraph{Adjacency and Laplacian.}
Define the adjacency operator $A_{G_n}$ and the degree operator $D_{G_n}$ on
$\ell^2(\mathcal D_n)$ by
\[
(A_{G_n}\psi)(w) := \sum_{w'\sim w} \psi(w'),
\qquad
(D_{G_n}\psi)(w) := \deg(w)\,\psi(w),
\]
where $w'\sim w$ denotes adjacency in $G_n$. The (combinatorial) graph Laplacian is
\[
\Delta_{G_n} := D_{G_n}-A_{G_n},
\]
and the associated diffusion generator is
\[
L_{G_n} := -\Delta_{G_n}.
\]
Then $\Delta_{G_n}$ is self-adjoint and positive semidefinite, while $L_{G_n}$ is
self-adjoint and negative semidefinite.

\paragraph{Discrete heat and Schr\"odinger equations.}
The heat equation on the tier graph is the linear ODE
\[
\partial_\tau u = L_{G_n}u,
\]
with solution $u(\tau)=e^{\tau L_{G_n}}u(0)$. Because $-\Delta_{G_n}$ is
self-adjoint and nonpositive, $e^{\tau L_{G_n}}$ is a contraction semigroup.
The corresponding unitary ``free'' Schr\"odinger evolution is
\[
i\,\partial_t \psi = -L_{G_n}\psi = \Delta_{G_n}\psi,
\]
with solution $\psi(t)=e^{-it\Delta_{G_n}}\psi(0)$.

\begin{remark}
This optional tier-graph framework does not fix a preferred choice of adjacency
$G_n$ and is not used elsewhere in the paper. Its purpose is to make explicit
that, once a tier-local notion of neighbourhood is specified, discrete diffusion
and Schr\"odinger-type evolutions on the Catalan state space follow by standard
graph-Laplacian constructions (compare Remark~\ref{rem:dyck-conditioned-drift-scaling}
and Remark~\ref{rem:dyck-conditioned-kernel-doob} for the tier-growth Markov
structure induced by Dyck conditioning).
\end{remark}

% ---------------------------------------------------------------------------
% Bibitems used only in this supplement (kept here for copy/paste convenience).
% If you re-`\input{supplemental-operators.tex}` into the main paper, these
% entries should live in the main `thebibliography` environment.
% ---------------------------------------------------------------------------
\iffalse
\begin{thebibliography}{99}

\bibitem{abhy2018scatteringforms}
N.~Arkani-Hamed, Y.~Bai, S.~He, and G.~Yan.
\newblock Scattering forms and the positive geometry of kinematics, color and
the worldsheet.
\newblock {\em JHEP} \textbf{05} (2018) 096.
\newblock arXiv:1711.09102.

\bibitem{banerjee2018stokes}
P.~Banerjee, A.~Laddha, and P.~Raman.
\newblock Stokes polytopes: The positive geometry for $\phi^4$ interactions.
\newblock arXiv:1811.05904, 2018.

\bibitem{herrmann2022positivegeometry}
E.~Herrmann and J.~Trnka.
\newblock Positive geometry of scattering amplitudes.
\newblock arXiv:2203.13018, 2022.

\end{thebibliography}
\fi
` into the main paper, these
% entries should live in the main `thebibliography` environment.
% ---------------------------------------------------------------------------
\iffalse
\begin{thebibliography}{99}

\bibitem{abhy2018scatteringforms}
N.~Arkani-Hamed, Y.~Bai, S.~He, and G.~Yan.
\newblock Scattering forms and the positive geometry of kinematics, color and
the worldsheet.
\newblock {\em JHEP} \textbf{05} (2018) 096.
\newblock arXiv:1711.09102.

\bibitem{banerjee2018stokes}
P.~Banerjee, A.~Laddha, and P.~Raman.
\newblock Stokes polytopes: The positive geometry for $\phi^4$ interactions.
\newblock arXiv:1811.05904, 2018.

\bibitem{herrmann2022positivegeometry}
E.~Herrmann and J.~Trnka.
\newblock Positive geometry of scattering amplitudes.
\newblock arXiv:2203.13018, 2022.

\end{thebibliography}
\fi
` into the main paper, these
% entries should live in the main `thebibliography` environment.
% ---------------------------------------------------------------------------
\iffalse
\begin{thebibliography}{99}

\bibitem{abhy2018scatteringforms}
N.~Arkani-Hamed, Y.~Bai, S.~He, and G.~Yan.
\newblock Scattering forms and the positive geometry of kinematics, color and
the worldsheet.
\newblock {\em JHEP} \textbf{05} (2018) 096.
\newblock arXiv:1711.09102.

\bibitem{banerjee2018stokes}
P.~Banerjee, A.~Laddha, and P.~Raman.
\newblock Stokes polytopes: The positive geometry for $\phi^4$ interactions.
\newblock arXiv:1811.05904, 2018.

\bibitem{herrmann2022positivegeometry}
E.~Herrmann and J.~Trnka.
\newblock Positive geometry of scattering amplitudes.
\newblock arXiv:2203.13018, 2022.

\end{thebibliography}
\fi
` at the former location in
% Appendix "Additional Technical Notes".

\subsection{Fields on words, prefixes, and nodes (optional)}
\label{subsec:fields}

We use the word ``field'' as shorthand for a complex-valued function on one of
the Catalan objects already in play. Several closely related state spaces are
useful in different contexts.

\paragraph{Fields on completed histories (fixed tier).}
Fix $n$ and consider a function $\Phi_n:\mathcal D_n\to\mathbb{C}$ assigning an
amplitude (or observable value) to each completed history $w\in\mathcal D_n$.
The associated Hilbert space is $\ell^2(\mathcal D_n)$ with inner product
\[
\langle \psi,\phi\rangle := \sum_{w\in\mathcal D_n} \overline{\psi(w)}\,\phi(w).
\]

\paragraph{Fields on prefixes (the full cone).}
Let $\mathcal{C}$ denote the set of Dyck prefixes (admissible partial histories).
A prefix field is a function $\Phi:\mathcal{C}\to\mathbb{C}$, which may be
restricted to a fixed length slice
$\mathcal{C}^{(k)}:=\{p\in\mathcal{C}: |p|=k\}$ when needed.

\paragraph{Fields on nodes of a fixed tree.}
Given $w\in\mathcal D_n$, let $T(w)$ be its associated full binary tree. A
node field is a function $\phi_w:\mathrm{Int}(T(w))\to\mathbb{C}$ on the internal
nodes of that tree.

\begin{remark}
These notions live on different objects (tiers, the prefix poset, or a single
tree) and are independent of any within-tier ordering convention on
$\mathcal D_n$.
\end{remark}

\subsection{Subtree indicators as a multiscale spanning family (optional)}
\label{subsec:subtree-indicators}

Let $T$ be a finite rooted tree and write $\mathrm{Int}(T)$ for its internal
nodes. Each $v\in\mathrm{Int}(T)$ determines a rooted subtree $T_v$, and hence a
subset $\mathrm{Int}(T_v)\subseteq \mathrm{Int}(T)$. Define the subtree indicator
\[
\chi_v:\mathrm{Int}(T)\to\{0,1\},
\qquad
\chi_v(u):=\mathbf{1}\{u\in \mathrm{Int}(T_v)\}.
\]

\begin{lemma}[Subtree indicators form a basis]
\label{lem:subtree-indicator-basis}
The family $\{\chi_v: v\in \mathrm{Int}(T)\}$ is a basis of the vector space of
complex-valued functions on $\mathrm{Int}(T)$.
\end{lemma}

\begin{proof}
Order the internal nodes by nonincreasing depth (deepest first), and let $M$ be
the square matrix with entries $M_{uv}:=\chi_v(u)$. Then $M_{vv}=1$ for all $v$,
while $M_{uv}=0$ whenever $u$ precedes $v$ in this order (a node cannot be a
descendant of a deeper node). Thus $M$ is triangular with ones on the diagonal,
hence invertible. Therefore the indicators are linearly independent and, since
their number equals $\#\mathrm{Int}(T)$, they form a basis.
\end{proof}

\begin{corollary}[Explicit inversion]
\label{cor:subtree-indicator-inversion}
Let $f:\mathrm{Int}(T)\to\mathbb{C}$ be any function. There is a unique family
of coefficients $\{a_v\}_{v\in\mathrm{Int}(T)}$ such that
\[
f \;=\; \sum_{v\in\mathrm{Int}(T)} a_v\,\chi_v.
\]
Writing $\mathrm{par}(v)$ for the parent of $v$ (for $v\neq \mathrm{root}(T)$),
these coefficients are given by
\[
a_{\mathrm{root}(T)} = f(\mathrm{root}(T)),
\qquad
a_v = f(v)-f(\mathrm{par}(v)) \quad (v\neq \mathrm{root}(T)).
\]
\end{corollary}

\begin{proof}
For each $u\in\mathrm{Int}(T)$,
$(\sum_v a_v\chi_v)(u)=\sum_{v:\,u\in\mathrm{Int}(T_v)} a_v
=\sum_{v\preceq u} a_v$, where $v\preceq u$ means that $v$ is an ancestor of
$u$. With the stated choice of coefficients, this ancestor sum telescopes along
the unique root-to-$u$ chain to yield $f(u)$. Uniqueness follows from
Lemma~\ref{lem:subtree-indicator-basis}.
\end{proof}

\begin{remark}
This basis is ``multiscale'': indicators of deep subtrees localize to fine
regions of $T$, while indicators near the root encode coarse structure. Any
choice of orthonormalization yields an orthonormal basis adapted to the rooted
tree geometry.
\end{remark}

\subsection{Operators on a fixed history tree (optional)}
\label{subsec:tree-operators}

In addition to tier-wise state spaces (fields on $\mathcal D_n$), one may also
consider dynamics \emph{within} a fixed realized history by placing operators on
the internal nodes of its tree.

\paragraph{Node Hilbert space.}
Fix $w\in\mathcal D_n$ and let $T(w)$ be its associated full binary tree. Write
$V_w:=\mathrm{Int}(T(w))$ and consider $\ell^2(V_w)$ with inner product
$\langle \psi,\phi\rangle := \sum_{v\in V_w}\overline{\psi(v)}\,\phi(v)$.

\paragraph{Adjacency and Laplacian.}
Let $G_w=(V_w,E_w)$ be any finite undirected graph on $V_w$ (for example, connect
each internal node to its internal children). Define $A_{G_w}$, $D_{G_w}$, and
the graph Laplacian and generator by
\[
\Delta_{G_w}:=D_{G_w}-A_{G_w},
\qquad
L_{G_w}:=-\Delta_{G_w}.
\]

\paragraph{Heat and Schr\"odinger evolutions.}
The corresponding ``internal-time'' heat equation is
\[
\partial_\tau u = L_{G_w}u,
\]
and the corresponding unitary Schr\"odinger evolution is
\[
i\,\partial_t \psi = -L_{G_w}\psi = \Delta_{G_w}\psi.
\]

\begin{remark}
This within-history operator framework is independent of the tier-growth Markov
dynamics and of coherent summation over histories: it simply records that, once
a graph structure is specified on the internal nodes of a fixed Catalan tree,
standard graph-Laplacian constructions yield discrete diffusion and
Schr\"odinger-type evolutions on that fixed combinatorial background.
\end{remark}

\subsection{Operators on tier slices (optional)}
\label{subsec:tier-operators}

The main text emphasizes two dynamics on the Catalan substrate: tier growth
(prefix extension) and coherent summation over histories. Independently, one may
also consider \emph{slice dynamics} on a fixed tier by endowing the finite set
$\mathcal D_n$ with an auxiliary adjacency graph. This subsection records the
standard operator framework for such constructions.

\paragraph{Tier Hilbert space.}
We take the tier state space to be $\ell^2(\mathcal D_n)$ as in
Section~\ref{subsec:fields}.

\paragraph{Adjacency graphs.}
Let $G_n=(\mathcal D_n,E_n)$ be any finite undirected graph on $\mathcal D_n$.
The choice of $G_n$ is additional structure: different graphs induce different
notions of locality on the tier. A canonical example is the rotation graph (the
associahedron adjacency) on full binary trees, where edges correspond to single
associativity rotations \cite{stanley-catalan,cheneviere2022linear}.

\paragraph{Associahedra and planar tree amplitudes (scattering-amplitude tie-in).}
The associahedron adjacency on $\mathcal D_n$ is also natural from the
perspective of scattering amplitudes. For the planar tree-level sector of
bi-adjoint cubic scalar theory ($\phi^3$), Arkani-Hamed, Bai, He, and Yan
identify an associahedron in planar kinematic space and show that the tree
amplitude is the corresponding canonical form of this positive geometry
\cite{abhy2018scatteringforms}. From this viewpoint, the Catalan enumeration of
planar cubic tree diagrams is not merely counting: the associahedron organizes
factorization channels geometrically, and different triangulations correspond to
different diagrammatic expansions of the same canonical form (see, e.g., the
review \cite{herrmann2022positivegeometry}). For quartic interactions, an
analogous positive-geometry description involves Stokes polytopes rather than
associahedra \cite{banerjee2018stokes}.

\paragraph{Tamari/Dyck/alt-Tamari choices on the same tier.}
The point of introducing an auxiliary graph $G_n$ is that the underlying state
set $\mathcal D_n$ supports multiple natural notions of tier-locality coming
from classical Catalan posets. The rotation graph is the undirected adjacency
underlying the Tamari order; one may likewise equip $\mathcal D_n$ with
adjacency induced by the Dyck (``Stanley'') lattice on Dyck paths, or more
generally by the family of $\delta$-Tamari (alt-Tamari) posets interpolating
between these extremes. These alternatives use different covering relations on
the same Catalan tier and therefore induce different graph Laplacians
$\Delta_{G_n}$ and different ``free'' tier Hamiltonians, but they live on a
common configuration space $\mathcal D_n$ \cite{stanley-catalan,cheneviere2022linear}.

\paragraph{Linear intervals as ``1D corridors'' in Catalan posets.}
A useful robustness fact is that certain one-dimensional substructures are
invariant across these Catalan posets: Chenevi\`ere proves that, for each fixed
$n$ and each height parameter $k$ (in the sense of \cite{cheneviere2022linear}),
the Tamari lattice and the Dyck lattice have the same number of \emph{linear
intervals} (intervals whose Hasse diagram is a chain), and moreover all
alt-Tamari posets share this same count at each height
\cite{cheneviere2022linear}. In the present language, this says that the number
of ``diamond-free corridors'' (regions with a unique maximal chain) is stable
under a wide class of tier-local adjacency choices, reinforcing the theme that
many distinct dynamics can be layered on a single Catalan substrate without
changing its most rigid combinatorial invariants.

\paragraph{Adjacency and Laplacian.}
Define the adjacency operator $A_{G_n}$ and the degree operator $D_{G_n}$ on
$\ell^2(\mathcal D_n)$ by
\[
(A_{G_n}\psi)(w) := \sum_{w'\sim w} \psi(w'),
\qquad
(D_{G_n}\psi)(w) := \deg(w)\,\psi(w),
\]
where $w'\sim w$ denotes adjacency in $G_n$. The (combinatorial) graph Laplacian is
\[
\Delta_{G_n} := D_{G_n}-A_{G_n},
\]
and the associated diffusion generator is
\[
L_{G_n} := -\Delta_{G_n}.
\]
Then $\Delta_{G_n}$ is self-adjoint and positive semidefinite, while $L_{G_n}$ is
self-adjoint and negative semidefinite.

\paragraph{Discrete heat and Schr\"odinger equations.}
The heat equation on the tier graph is the linear ODE
\[
\partial_\tau u = L_{G_n}u,
\]
with solution $u(\tau)=e^{\tau L_{G_n}}u(0)$. Because $-\Delta_{G_n}$ is
self-adjoint and nonpositive, $e^{\tau L_{G_n}}$ is a contraction semigroup.
The corresponding unitary ``free'' Schr\"odinger evolution is
\[
i\,\partial_t \psi = -L_{G_n}\psi = \Delta_{G_n}\psi,
\]
with solution $\psi(t)=e^{-it\Delta_{G_n}}\psi(0)$.

\begin{remark}
This optional tier-graph framework does not fix a preferred choice of adjacency
$G_n$ and is not used elsewhere in the paper. Its purpose is to make explicit
that, once a tier-local notion of neighbourhood is specified, discrete diffusion
and Schr\"odinger-type evolutions on the Catalan state space follow by standard
graph-Laplacian constructions (compare Remark~\ref{rem:dyck-conditioned-drift-scaling}
and Remark~\ref{rem:dyck-conditioned-kernel-doob} for the tier-growth Markov
structure induced by Dyck conditioning).
\end{remark}

% ---------------------------------------------------------------------------
% Bibitems used only in this supplement (kept here for copy/paste convenience).
% If you re-`% Supplemental material extracted from `docs/catalan-light-cone.tex`.
% This block was removed from the compiled arXiv PDF to keep the main v1 lean.
% Intended use: `\% Supplemental material extracted from `docs/catalan-light-cone.tex`.
% This block was removed from the compiled arXiv PDF to keep the main v1 lean.
% Intended use: `\% Supplemental material extracted from `docs/catalan-light-cone.tex`.
% This block was removed from the compiled arXiv PDF to keep the main v1 lean.
% Intended use: `\\input{supplemental-operators.tex}` at the former location in
% Appendix "Additional Technical Notes".

\subsection{Fields on words, prefixes, and nodes (optional)}
\label{subsec:fields}

We use the word ``field'' as shorthand for a complex-valued function on one of
the Catalan objects already in play. Several closely related state spaces are
useful in different contexts.

\paragraph{Fields on completed histories (fixed tier).}
Fix $n$ and consider a function $\Phi_n:\mathcal D_n\to\mathbb{C}$ assigning an
amplitude (or observable value) to each completed history $w\in\mathcal D_n$.
The associated Hilbert space is $\ell^2(\mathcal D_n)$ with inner product
\[
\langle \psi,\phi\rangle := \sum_{w\in\mathcal D_n} \overline{\psi(w)}\,\phi(w).
\]

\paragraph{Fields on prefixes (the full cone).}
Let $\mathcal{C}$ denote the set of Dyck prefixes (admissible partial histories).
A prefix field is a function $\Phi:\mathcal{C}\to\mathbb{C}$, which may be
restricted to a fixed length slice
$\mathcal{C}^{(k)}:=\{p\in\mathcal{C}: |p|=k\}$ when needed.

\paragraph{Fields on nodes of a fixed tree.}
Given $w\in\mathcal D_n$, let $T(w)$ be its associated full binary tree. A
node field is a function $\phi_w:\mathrm{Int}(T(w))\to\mathbb{C}$ on the internal
nodes of that tree.

\begin{remark}
These notions live on different objects (tiers, the prefix poset, or a single
tree) and are independent of any within-tier ordering convention on
$\mathcal D_n$.
\end{remark}

\subsection{Subtree indicators as a multiscale spanning family (optional)}
\label{subsec:subtree-indicators}

Let $T$ be a finite rooted tree and write $\mathrm{Int}(T)$ for its internal
nodes. Each $v\in\mathrm{Int}(T)$ determines a rooted subtree $T_v$, and hence a
subset $\mathrm{Int}(T_v)\subseteq \mathrm{Int}(T)$. Define the subtree indicator
\[
\chi_v:\mathrm{Int}(T)\to\{0,1\},
\qquad
\chi_v(u):=\mathbf{1}\{u\in \mathrm{Int}(T_v)\}.
\]

\begin{lemma}[Subtree indicators form a basis]
\label{lem:subtree-indicator-basis}
The family $\{\chi_v: v\in \mathrm{Int}(T)\}$ is a basis of the vector space of
complex-valued functions on $\mathrm{Int}(T)$.
\end{lemma}

\begin{proof}
Order the internal nodes by nonincreasing depth (deepest first), and let $M$ be
the square matrix with entries $M_{uv}:=\chi_v(u)$. Then $M_{vv}=1$ for all $v$,
while $M_{uv}=0$ whenever $u$ precedes $v$ in this order (a node cannot be a
descendant of a deeper node). Thus $M$ is triangular with ones on the diagonal,
hence invertible. Therefore the indicators are linearly independent and, since
their number equals $\#\mathrm{Int}(T)$, they form a basis.
\end{proof}

\begin{corollary}[Explicit inversion]
\label{cor:subtree-indicator-inversion}
Let $f:\mathrm{Int}(T)\to\mathbb{C}$ be any function. There is a unique family
of coefficients $\{a_v\}_{v\in\mathrm{Int}(T)}$ such that
\[
f \;=\; \sum_{v\in\mathrm{Int}(T)} a_v\,\chi_v.
\]
Writing $\mathrm{par}(v)$ for the parent of $v$ (for $v\neq \mathrm{root}(T)$),
these coefficients are given by
\[
a_{\mathrm{root}(T)} = f(\mathrm{root}(T)),
\qquad
a_v = f(v)-f(\mathrm{par}(v)) \quad (v\neq \mathrm{root}(T)).
\]
\end{corollary}

\begin{proof}
For each $u\in\mathrm{Int}(T)$,
$(\sum_v a_v\chi_v)(u)=\sum_{v:\,u\in\mathrm{Int}(T_v)} a_v
=\sum_{v\preceq u} a_v$, where $v\preceq u$ means that $v$ is an ancestor of
$u$. With the stated choice of coefficients, this ancestor sum telescopes along
the unique root-to-$u$ chain to yield $f(u)$. Uniqueness follows from
Lemma~\ref{lem:subtree-indicator-basis}.
\end{proof}

\begin{remark}
This basis is ``multiscale'': indicators of deep subtrees localize to fine
regions of $T$, while indicators near the root encode coarse structure. Any
choice of orthonormalization yields an orthonormal basis adapted to the rooted
tree geometry.
\end{remark}

\subsection{Operators on a fixed history tree (optional)}
\label{subsec:tree-operators}

In addition to tier-wise state spaces (fields on $\mathcal D_n$), one may also
consider dynamics \emph{within} a fixed realized history by placing operators on
the internal nodes of its tree.

\paragraph{Node Hilbert space.}
Fix $w\in\mathcal D_n$ and let $T(w)$ be its associated full binary tree. Write
$V_w:=\mathrm{Int}(T(w))$ and consider $\ell^2(V_w)$ with inner product
$\langle \psi,\phi\rangle := \sum_{v\in V_w}\overline{\psi(v)}\,\phi(v)$.

\paragraph{Adjacency and Laplacian.}
Let $G_w=(V_w,E_w)$ be any finite undirected graph on $V_w$ (for example, connect
each internal node to its internal children). Define $A_{G_w}$, $D_{G_w}$, and
the graph Laplacian and generator by
\[
\Delta_{G_w}:=D_{G_w}-A_{G_w},
\qquad
L_{G_w}:=-\Delta_{G_w}.
\]

\paragraph{Heat and Schr\"odinger evolutions.}
The corresponding ``internal-time'' heat equation is
\[
\partial_\tau u = L_{G_w}u,
\]
and the corresponding unitary Schr\"odinger evolution is
\[
i\,\partial_t \psi = -L_{G_w}\psi = \Delta_{G_w}\psi.
\]

\begin{remark}
This within-history operator framework is independent of the tier-growth Markov
dynamics and of coherent summation over histories: it simply records that, once
a graph structure is specified on the internal nodes of a fixed Catalan tree,
standard graph-Laplacian constructions yield discrete diffusion and
Schr\"odinger-type evolutions on that fixed combinatorial background.
\end{remark}

\subsection{Operators on tier slices (optional)}
\label{subsec:tier-operators}

The main text emphasizes two dynamics on the Catalan substrate: tier growth
(prefix extension) and coherent summation over histories. Independently, one may
also consider \emph{slice dynamics} on a fixed tier by endowing the finite set
$\mathcal D_n$ with an auxiliary adjacency graph. This subsection records the
standard operator framework for such constructions.

\paragraph{Tier Hilbert space.}
We take the tier state space to be $\ell^2(\mathcal D_n)$ as in
Section~\ref{subsec:fields}.

\paragraph{Adjacency graphs.}
Let $G_n=(\mathcal D_n,E_n)$ be any finite undirected graph on $\mathcal D_n$.
The choice of $G_n$ is additional structure: different graphs induce different
notions of locality on the tier. A canonical example is the rotation graph (the
associahedron adjacency) on full binary trees, where edges correspond to single
associativity rotations \cite{stanley-catalan,cheneviere2022linear}.

\paragraph{Associahedra and planar tree amplitudes (scattering-amplitude tie-in).}
The associahedron adjacency on $\mathcal D_n$ is also natural from the
perspective of scattering amplitudes. For the planar tree-level sector of
bi-adjoint cubic scalar theory ($\phi^3$), Arkani-Hamed, Bai, He, and Yan
identify an associahedron in planar kinematic space and show that the tree
amplitude is the corresponding canonical form of this positive geometry
\cite{abhy2018scatteringforms}. From this viewpoint, the Catalan enumeration of
planar cubic tree diagrams is not merely counting: the associahedron organizes
factorization channels geometrically, and different triangulations correspond to
different diagrammatic expansions of the same canonical form (see, e.g., the
review \cite{herrmann2022positivegeometry}). For quartic interactions, an
analogous positive-geometry description involves Stokes polytopes rather than
associahedra \cite{banerjee2018stokes}.

\paragraph{Tamari/Dyck/alt-Tamari choices on the same tier.}
The point of introducing an auxiliary graph $G_n$ is that the underlying state
set $\mathcal D_n$ supports multiple natural notions of tier-locality coming
from classical Catalan posets. The rotation graph is the undirected adjacency
underlying the Tamari order; one may likewise equip $\mathcal D_n$ with
adjacency induced by the Dyck (``Stanley'') lattice on Dyck paths, or more
generally by the family of $\delta$-Tamari (alt-Tamari) posets interpolating
between these extremes. These alternatives use different covering relations on
the same Catalan tier and therefore induce different graph Laplacians
$\Delta_{G_n}$ and different ``free'' tier Hamiltonians, but they live on a
common configuration space $\mathcal D_n$ \cite{stanley-catalan,cheneviere2022linear}.

\paragraph{Linear intervals as ``1D corridors'' in Catalan posets.}
A useful robustness fact is that certain one-dimensional substructures are
invariant across these Catalan posets: Chenevi\`ere proves that, for each fixed
$n$ and each height parameter $k$ (in the sense of \cite{cheneviere2022linear}),
the Tamari lattice and the Dyck lattice have the same number of \emph{linear
intervals} (intervals whose Hasse diagram is a chain), and moreover all
alt-Tamari posets share this same count at each height
\cite{cheneviere2022linear}. In the present language, this says that the number
of ``diamond-free corridors'' (regions with a unique maximal chain) is stable
under a wide class of tier-local adjacency choices, reinforcing the theme that
many distinct dynamics can be layered on a single Catalan substrate without
changing its most rigid combinatorial invariants.

\paragraph{Adjacency and Laplacian.}
Define the adjacency operator $A_{G_n}$ and the degree operator $D_{G_n}$ on
$\ell^2(\mathcal D_n)$ by
\[
(A_{G_n}\psi)(w) := \sum_{w'\sim w} \psi(w'),
\qquad
(D_{G_n}\psi)(w) := \deg(w)\,\psi(w),
\]
where $w'\sim w$ denotes adjacency in $G_n$. The (combinatorial) graph Laplacian is
\[
\Delta_{G_n} := D_{G_n}-A_{G_n},
\]
and the associated diffusion generator is
\[
L_{G_n} := -\Delta_{G_n}.
\]
Then $\Delta_{G_n}$ is self-adjoint and positive semidefinite, while $L_{G_n}$ is
self-adjoint and negative semidefinite.

\paragraph{Discrete heat and Schr\"odinger equations.}
The heat equation on the tier graph is the linear ODE
\[
\partial_\tau u = L_{G_n}u,
\]
with solution $u(\tau)=e^{\tau L_{G_n}}u(0)$. Because $-\Delta_{G_n}$ is
self-adjoint and nonpositive, $e^{\tau L_{G_n}}$ is a contraction semigroup.
The corresponding unitary ``free'' Schr\"odinger evolution is
\[
i\,\partial_t \psi = -L_{G_n}\psi = \Delta_{G_n}\psi,
\]
with solution $\psi(t)=e^{-it\Delta_{G_n}}\psi(0)$.

\begin{remark}
This optional tier-graph framework does not fix a preferred choice of adjacency
$G_n$ and is not used elsewhere in the paper. Its purpose is to make explicit
that, once a tier-local notion of neighbourhood is specified, discrete diffusion
and Schr\"odinger-type evolutions on the Catalan state space follow by standard
graph-Laplacian constructions (compare Remark~\ref{rem:dyck-conditioned-drift-scaling}
and Remark~\ref{rem:dyck-conditioned-kernel-doob} for the tier-growth Markov
structure induced by Dyck conditioning).
\end{remark}

% ---------------------------------------------------------------------------
% Bibitems used only in this supplement (kept here for copy/paste convenience).
% If you re-`\input{supplemental-operators.tex}` into the main paper, these
% entries should live in the main `thebibliography` environment.
% ---------------------------------------------------------------------------
\iffalse
\begin{thebibliography}{99}

\bibitem{abhy2018scatteringforms}
N.~Arkani-Hamed, Y.~Bai, S.~He, and G.~Yan.
\newblock Scattering forms and the positive geometry of kinematics, color and
the worldsheet.
\newblock {\em JHEP} \textbf{05} (2018) 096.
\newblock arXiv:1711.09102.

\bibitem{banerjee2018stokes}
P.~Banerjee, A.~Laddha, and P.~Raman.
\newblock Stokes polytopes: The positive geometry for $\phi^4$ interactions.
\newblock arXiv:1811.05904, 2018.

\bibitem{herrmann2022positivegeometry}
E.~Herrmann and J.~Trnka.
\newblock Positive geometry of scattering amplitudes.
\newblock arXiv:2203.13018, 2022.

\end{thebibliography}
\fi
` at the former location in
% Appendix "Additional Technical Notes".

\subsection{Fields on words, prefixes, and nodes (optional)}
\label{subsec:fields}

We use the word ``field'' as shorthand for a complex-valued function on one of
the Catalan objects already in play. Several closely related state spaces are
useful in different contexts.

\paragraph{Fields on completed histories (fixed tier).}
Fix $n$ and consider a function $\Phi_n:\mathcal D_n\to\mathbb{C}$ assigning an
amplitude (or observable value) to each completed history $w\in\mathcal D_n$.
The associated Hilbert space is $\ell^2(\mathcal D_n)$ with inner product
\[
\langle \psi,\phi\rangle := \sum_{w\in\mathcal D_n} \overline{\psi(w)}\,\phi(w).
\]

\paragraph{Fields on prefixes (the full cone).}
Let $\mathcal{C}$ denote the set of Dyck prefixes (admissible partial histories).
A prefix field is a function $\Phi:\mathcal{C}\to\mathbb{C}$, which may be
restricted to a fixed length slice
$\mathcal{C}^{(k)}:=\{p\in\mathcal{C}: |p|=k\}$ when needed.

\paragraph{Fields on nodes of a fixed tree.}
Given $w\in\mathcal D_n$, let $T(w)$ be its associated full binary tree. A
node field is a function $\phi_w:\mathrm{Int}(T(w))\to\mathbb{C}$ on the internal
nodes of that tree.

\begin{remark}
These notions live on different objects (tiers, the prefix poset, or a single
tree) and are independent of any within-tier ordering convention on
$\mathcal D_n$.
\end{remark}

\subsection{Subtree indicators as a multiscale spanning family (optional)}
\label{subsec:subtree-indicators}

Let $T$ be a finite rooted tree and write $\mathrm{Int}(T)$ for its internal
nodes. Each $v\in\mathrm{Int}(T)$ determines a rooted subtree $T_v$, and hence a
subset $\mathrm{Int}(T_v)\subseteq \mathrm{Int}(T)$. Define the subtree indicator
\[
\chi_v:\mathrm{Int}(T)\to\{0,1\},
\qquad
\chi_v(u):=\mathbf{1}\{u\in \mathrm{Int}(T_v)\}.
\]

\begin{lemma}[Subtree indicators form a basis]
\label{lem:subtree-indicator-basis}
The family $\{\chi_v: v\in \mathrm{Int}(T)\}$ is a basis of the vector space of
complex-valued functions on $\mathrm{Int}(T)$.
\end{lemma}

\begin{proof}
Order the internal nodes by nonincreasing depth (deepest first), and let $M$ be
the square matrix with entries $M_{uv}:=\chi_v(u)$. Then $M_{vv}=1$ for all $v$,
while $M_{uv}=0$ whenever $u$ precedes $v$ in this order (a node cannot be a
descendant of a deeper node). Thus $M$ is triangular with ones on the diagonal,
hence invertible. Therefore the indicators are linearly independent and, since
their number equals $\#\mathrm{Int}(T)$, they form a basis.
\end{proof}

\begin{corollary}[Explicit inversion]
\label{cor:subtree-indicator-inversion}
Let $f:\mathrm{Int}(T)\to\mathbb{C}$ be any function. There is a unique family
of coefficients $\{a_v\}_{v\in\mathrm{Int}(T)}$ such that
\[
f \;=\; \sum_{v\in\mathrm{Int}(T)} a_v\,\chi_v.
\]
Writing $\mathrm{par}(v)$ for the parent of $v$ (for $v\neq \mathrm{root}(T)$),
these coefficients are given by
\[
a_{\mathrm{root}(T)} = f(\mathrm{root}(T)),
\qquad
a_v = f(v)-f(\mathrm{par}(v)) \quad (v\neq \mathrm{root}(T)).
\]
\end{corollary}

\begin{proof}
For each $u\in\mathrm{Int}(T)$,
$(\sum_v a_v\chi_v)(u)=\sum_{v:\,u\in\mathrm{Int}(T_v)} a_v
=\sum_{v\preceq u} a_v$, where $v\preceq u$ means that $v$ is an ancestor of
$u$. With the stated choice of coefficients, this ancestor sum telescopes along
the unique root-to-$u$ chain to yield $f(u)$. Uniqueness follows from
Lemma~\ref{lem:subtree-indicator-basis}.
\end{proof}

\begin{remark}
This basis is ``multiscale'': indicators of deep subtrees localize to fine
regions of $T$, while indicators near the root encode coarse structure. Any
choice of orthonormalization yields an orthonormal basis adapted to the rooted
tree geometry.
\end{remark}

\subsection{Operators on a fixed history tree (optional)}
\label{subsec:tree-operators}

In addition to tier-wise state spaces (fields on $\mathcal D_n$), one may also
consider dynamics \emph{within} a fixed realized history by placing operators on
the internal nodes of its tree.

\paragraph{Node Hilbert space.}
Fix $w\in\mathcal D_n$ and let $T(w)$ be its associated full binary tree. Write
$V_w:=\mathrm{Int}(T(w))$ and consider $\ell^2(V_w)$ with inner product
$\langle \psi,\phi\rangle := \sum_{v\in V_w}\overline{\psi(v)}\,\phi(v)$.

\paragraph{Adjacency and Laplacian.}
Let $G_w=(V_w,E_w)$ be any finite undirected graph on $V_w$ (for example, connect
each internal node to its internal children). Define $A_{G_w}$, $D_{G_w}$, and
the graph Laplacian and generator by
\[
\Delta_{G_w}:=D_{G_w}-A_{G_w},
\qquad
L_{G_w}:=-\Delta_{G_w}.
\]

\paragraph{Heat and Schr\"odinger evolutions.}
The corresponding ``internal-time'' heat equation is
\[
\partial_\tau u = L_{G_w}u,
\]
and the corresponding unitary Schr\"odinger evolution is
\[
i\,\partial_t \psi = -L_{G_w}\psi = \Delta_{G_w}\psi.
\]

\begin{remark}
This within-history operator framework is independent of the tier-growth Markov
dynamics and of coherent summation over histories: it simply records that, once
a graph structure is specified on the internal nodes of a fixed Catalan tree,
standard graph-Laplacian constructions yield discrete diffusion and
Schr\"odinger-type evolutions on that fixed combinatorial background.
\end{remark}

\subsection{Operators on tier slices (optional)}
\label{subsec:tier-operators}

The main text emphasizes two dynamics on the Catalan substrate: tier growth
(prefix extension) and coherent summation over histories. Independently, one may
also consider \emph{slice dynamics} on a fixed tier by endowing the finite set
$\mathcal D_n$ with an auxiliary adjacency graph. This subsection records the
standard operator framework for such constructions.

\paragraph{Tier Hilbert space.}
We take the tier state space to be $\ell^2(\mathcal D_n)$ as in
Section~\ref{subsec:fields}.

\paragraph{Adjacency graphs.}
Let $G_n=(\mathcal D_n,E_n)$ be any finite undirected graph on $\mathcal D_n$.
The choice of $G_n$ is additional structure: different graphs induce different
notions of locality on the tier. A canonical example is the rotation graph (the
associahedron adjacency) on full binary trees, where edges correspond to single
associativity rotations \cite{stanley-catalan,cheneviere2022linear}.

\paragraph{Associahedra and planar tree amplitudes (scattering-amplitude tie-in).}
The associahedron adjacency on $\mathcal D_n$ is also natural from the
perspective of scattering amplitudes. For the planar tree-level sector of
bi-adjoint cubic scalar theory ($\phi^3$), Arkani-Hamed, Bai, He, and Yan
identify an associahedron in planar kinematic space and show that the tree
amplitude is the corresponding canonical form of this positive geometry
\cite{abhy2018scatteringforms}. From this viewpoint, the Catalan enumeration of
planar cubic tree diagrams is not merely counting: the associahedron organizes
factorization channels geometrically, and different triangulations correspond to
different diagrammatic expansions of the same canonical form (see, e.g., the
review \cite{herrmann2022positivegeometry}). For quartic interactions, an
analogous positive-geometry description involves Stokes polytopes rather than
associahedra \cite{banerjee2018stokes}.

\paragraph{Tamari/Dyck/alt-Tamari choices on the same tier.}
The point of introducing an auxiliary graph $G_n$ is that the underlying state
set $\mathcal D_n$ supports multiple natural notions of tier-locality coming
from classical Catalan posets. The rotation graph is the undirected adjacency
underlying the Tamari order; one may likewise equip $\mathcal D_n$ with
adjacency induced by the Dyck (``Stanley'') lattice on Dyck paths, or more
generally by the family of $\delta$-Tamari (alt-Tamari) posets interpolating
between these extremes. These alternatives use different covering relations on
the same Catalan tier and therefore induce different graph Laplacians
$\Delta_{G_n}$ and different ``free'' tier Hamiltonians, but they live on a
common configuration space $\mathcal D_n$ \cite{stanley-catalan,cheneviere2022linear}.

\paragraph{Linear intervals as ``1D corridors'' in Catalan posets.}
A useful robustness fact is that certain one-dimensional substructures are
invariant across these Catalan posets: Chenevi\`ere proves that, for each fixed
$n$ and each height parameter $k$ (in the sense of \cite{cheneviere2022linear}),
the Tamari lattice and the Dyck lattice have the same number of \emph{linear
intervals} (intervals whose Hasse diagram is a chain), and moreover all
alt-Tamari posets share this same count at each height
\cite{cheneviere2022linear}. In the present language, this says that the number
of ``diamond-free corridors'' (regions with a unique maximal chain) is stable
under a wide class of tier-local adjacency choices, reinforcing the theme that
many distinct dynamics can be layered on a single Catalan substrate without
changing its most rigid combinatorial invariants.

\paragraph{Adjacency and Laplacian.}
Define the adjacency operator $A_{G_n}$ and the degree operator $D_{G_n}$ on
$\ell^2(\mathcal D_n)$ by
\[
(A_{G_n}\psi)(w) := \sum_{w'\sim w} \psi(w'),
\qquad
(D_{G_n}\psi)(w) := \deg(w)\,\psi(w),
\]
where $w'\sim w$ denotes adjacency in $G_n$. The (combinatorial) graph Laplacian is
\[
\Delta_{G_n} := D_{G_n}-A_{G_n},
\]
and the associated diffusion generator is
\[
L_{G_n} := -\Delta_{G_n}.
\]
Then $\Delta_{G_n}$ is self-adjoint and positive semidefinite, while $L_{G_n}$ is
self-adjoint and negative semidefinite.

\paragraph{Discrete heat and Schr\"odinger equations.}
The heat equation on the tier graph is the linear ODE
\[
\partial_\tau u = L_{G_n}u,
\]
with solution $u(\tau)=e^{\tau L_{G_n}}u(0)$. Because $-\Delta_{G_n}$ is
self-adjoint and nonpositive, $e^{\tau L_{G_n}}$ is a contraction semigroup.
The corresponding unitary ``free'' Schr\"odinger evolution is
\[
i\,\partial_t \psi = -L_{G_n}\psi = \Delta_{G_n}\psi,
\]
with solution $\psi(t)=e^{-it\Delta_{G_n}}\psi(0)$.

\begin{remark}
This optional tier-graph framework does not fix a preferred choice of adjacency
$G_n$ and is not used elsewhere in the paper. Its purpose is to make explicit
that, once a tier-local notion of neighbourhood is specified, discrete diffusion
and Schr\"odinger-type evolutions on the Catalan state space follow by standard
graph-Laplacian constructions (compare Remark~\ref{rem:dyck-conditioned-drift-scaling}
and Remark~\ref{rem:dyck-conditioned-kernel-doob} for the tier-growth Markov
structure induced by Dyck conditioning).
\end{remark}

% ---------------------------------------------------------------------------
% Bibitems used only in this supplement (kept here for copy/paste convenience).
% If you re-`% Supplemental material extracted from `docs/catalan-light-cone.tex`.
% This block was removed from the compiled arXiv PDF to keep the main v1 lean.
% Intended use: `\\input{supplemental-operators.tex}` at the former location in
% Appendix "Additional Technical Notes".

\subsection{Fields on words, prefixes, and nodes (optional)}
\label{subsec:fields}

We use the word ``field'' as shorthand for a complex-valued function on one of
the Catalan objects already in play. Several closely related state spaces are
useful in different contexts.

\paragraph{Fields on completed histories (fixed tier).}
Fix $n$ and consider a function $\Phi_n:\mathcal D_n\to\mathbb{C}$ assigning an
amplitude (or observable value) to each completed history $w\in\mathcal D_n$.
The associated Hilbert space is $\ell^2(\mathcal D_n)$ with inner product
\[
\langle \psi,\phi\rangle := \sum_{w\in\mathcal D_n} \overline{\psi(w)}\,\phi(w).
\]

\paragraph{Fields on prefixes (the full cone).}
Let $\mathcal{C}$ denote the set of Dyck prefixes (admissible partial histories).
A prefix field is a function $\Phi:\mathcal{C}\to\mathbb{C}$, which may be
restricted to a fixed length slice
$\mathcal{C}^{(k)}:=\{p\in\mathcal{C}: |p|=k\}$ when needed.

\paragraph{Fields on nodes of a fixed tree.}
Given $w\in\mathcal D_n$, let $T(w)$ be its associated full binary tree. A
node field is a function $\phi_w:\mathrm{Int}(T(w))\to\mathbb{C}$ on the internal
nodes of that tree.

\begin{remark}
These notions live on different objects (tiers, the prefix poset, or a single
tree) and are independent of any within-tier ordering convention on
$\mathcal D_n$.
\end{remark}

\subsection{Subtree indicators as a multiscale spanning family (optional)}
\label{subsec:subtree-indicators}

Let $T$ be a finite rooted tree and write $\mathrm{Int}(T)$ for its internal
nodes. Each $v\in\mathrm{Int}(T)$ determines a rooted subtree $T_v$, and hence a
subset $\mathrm{Int}(T_v)\subseteq \mathrm{Int}(T)$. Define the subtree indicator
\[
\chi_v:\mathrm{Int}(T)\to\{0,1\},
\qquad
\chi_v(u):=\mathbf{1}\{u\in \mathrm{Int}(T_v)\}.
\]

\begin{lemma}[Subtree indicators form a basis]
\label{lem:subtree-indicator-basis}
The family $\{\chi_v: v\in \mathrm{Int}(T)\}$ is a basis of the vector space of
complex-valued functions on $\mathrm{Int}(T)$.
\end{lemma}

\begin{proof}
Order the internal nodes by nonincreasing depth (deepest first), and let $M$ be
the square matrix with entries $M_{uv}:=\chi_v(u)$. Then $M_{vv}=1$ for all $v$,
while $M_{uv}=0$ whenever $u$ precedes $v$ in this order (a node cannot be a
descendant of a deeper node). Thus $M$ is triangular with ones on the diagonal,
hence invertible. Therefore the indicators are linearly independent and, since
their number equals $\#\mathrm{Int}(T)$, they form a basis.
\end{proof}

\begin{corollary}[Explicit inversion]
\label{cor:subtree-indicator-inversion}
Let $f:\mathrm{Int}(T)\to\mathbb{C}$ be any function. There is a unique family
of coefficients $\{a_v\}_{v\in\mathrm{Int}(T)}$ such that
\[
f \;=\; \sum_{v\in\mathrm{Int}(T)} a_v\,\chi_v.
\]
Writing $\mathrm{par}(v)$ for the parent of $v$ (for $v\neq \mathrm{root}(T)$),
these coefficients are given by
\[
a_{\mathrm{root}(T)} = f(\mathrm{root}(T)),
\qquad
a_v = f(v)-f(\mathrm{par}(v)) \quad (v\neq \mathrm{root}(T)).
\]
\end{corollary}

\begin{proof}
For each $u\in\mathrm{Int}(T)$,
$(\sum_v a_v\chi_v)(u)=\sum_{v:\,u\in\mathrm{Int}(T_v)} a_v
=\sum_{v\preceq u} a_v$, where $v\preceq u$ means that $v$ is an ancestor of
$u$. With the stated choice of coefficients, this ancestor sum telescopes along
the unique root-to-$u$ chain to yield $f(u)$. Uniqueness follows from
Lemma~\ref{lem:subtree-indicator-basis}.
\end{proof}

\begin{remark}
This basis is ``multiscale'': indicators of deep subtrees localize to fine
regions of $T$, while indicators near the root encode coarse structure. Any
choice of orthonormalization yields an orthonormal basis adapted to the rooted
tree geometry.
\end{remark}

\subsection{Operators on a fixed history tree (optional)}
\label{subsec:tree-operators}

In addition to tier-wise state spaces (fields on $\mathcal D_n$), one may also
consider dynamics \emph{within} a fixed realized history by placing operators on
the internal nodes of its tree.

\paragraph{Node Hilbert space.}
Fix $w\in\mathcal D_n$ and let $T(w)$ be its associated full binary tree. Write
$V_w:=\mathrm{Int}(T(w))$ and consider $\ell^2(V_w)$ with inner product
$\langle \psi,\phi\rangle := \sum_{v\in V_w}\overline{\psi(v)}\,\phi(v)$.

\paragraph{Adjacency and Laplacian.}
Let $G_w=(V_w,E_w)$ be any finite undirected graph on $V_w$ (for example, connect
each internal node to its internal children). Define $A_{G_w}$, $D_{G_w}$, and
the graph Laplacian and generator by
\[
\Delta_{G_w}:=D_{G_w}-A_{G_w},
\qquad
L_{G_w}:=-\Delta_{G_w}.
\]

\paragraph{Heat and Schr\"odinger evolutions.}
The corresponding ``internal-time'' heat equation is
\[
\partial_\tau u = L_{G_w}u,
\]
and the corresponding unitary Schr\"odinger evolution is
\[
i\,\partial_t \psi = -L_{G_w}\psi = \Delta_{G_w}\psi.
\]

\begin{remark}
This within-history operator framework is independent of the tier-growth Markov
dynamics and of coherent summation over histories: it simply records that, once
a graph structure is specified on the internal nodes of a fixed Catalan tree,
standard graph-Laplacian constructions yield discrete diffusion and
Schr\"odinger-type evolutions on that fixed combinatorial background.
\end{remark}

\subsection{Operators on tier slices (optional)}
\label{subsec:tier-operators}

The main text emphasizes two dynamics on the Catalan substrate: tier growth
(prefix extension) and coherent summation over histories. Independently, one may
also consider \emph{slice dynamics} on a fixed tier by endowing the finite set
$\mathcal D_n$ with an auxiliary adjacency graph. This subsection records the
standard operator framework for such constructions.

\paragraph{Tier Hilbert space.}
We take the tier state space to be $\ell^2(\mathcal D_n)$ as in
Section~\ref{subsec:fields}.

\paragraph{Adjacency graphs.}
Let $G_n=(\mathcal D_n,E_n)$ be any finite undirected graph on $\mathcal D_n$.
The choice of $G_n$ is additional structure: different graphs induce different
notions of locality on the tier. A canonical example is the rotation graph (the
associahedron adjacency) on full binary trees, where edges correspond to single
associativity rotations \cite{stanley-catalan,cheneviere2022linear}.

\paragraph{Associahedra and planar tree amplitudes (scattering-amplitude tie-in).}
The associahedron adjacency on $\mathcal D_n$ is also natural from the
perspective of scattering amplitudes. For the planar tree-level sector of
bi-adjoint cubic scalar theory ($\phi^3$), Arkani-Hamed, Bai, He, and Yan
identify an associahedron in planar kinematic space and show that the tree
amplitude is the corresponding canonical form of this positive geometry
\cite{abhy2018scatteringforms}. From this viewpoint, the Catalan enumeration of
planar cubic tree diagrams is not merely counting: the associahedron organizes
factorization channels geometrically, and different triangulations correspond to
different diagrammatic expansions of the same canonical form (see, e.g., the
review \cite{herrmann2022positivegeometry}). For quartic interactions, an
analogous positive-geometry description involves Stokes polytopes rather than
associahedra \cite{banerjee2018stokes}.

\paragraph{Tamari/Dyck/alt-Tamari choices on the same tier.}
The point of introducing an auxiliary graph $G_n$ is that the underlying state
set $\mathcal D_n$ supports multiple natural notions of tier-locality coming
from classical Catalan posets. The rotation graph is the undirected adjacency
underlying the Tamari order; one may likewise equip $\mathcal D_n$ with
adjacency induced by the Dyck (``Stanley'') lattice on Dyck paths, or more
generally by the family of $\delta$-Tamari (alt-Tamari) posets interpolating
between these extremes. These alternatives use different covering relations on
the same Catalan tier and therefore induce different graph Laplacians
$\Delta_{G_n}$ and different ``free'' tier Hamiltonians, but they live on a
common configuration space $\mathcal D_n$ \cite{stanley-catalan,cheneviere2022linear}.

\paragraph{Linear intervals as ``1D corridors'' in Catalan posets.}
A useful robustness fact is that certain one-dimensional substructures are
invariant across these Catalan posets: Chenevi\`ere proves that, for each fixed
$n$ and each height parameter $k$ (in the sense of \cite{cheneviere2022linear}),
the Tamari lattice and the Dyck lattice have the same number of \emph{linear
intervals} (intervals whose Hasse diagram is a chain), and moreover all
alt-Tamari posets share this same count at each height
\cite{cheneviere2022linear}. In the present language, this says that the number
of ``diamond-free corridors'' (regions with a unique maximal chain) is stable
under a wide class of tier-local adjacency choices, reinforcing the theme that
many distinct dynamics can be layered on a single Catalan substrate without
changing its most rigid combinatorial invariants.

\paragraph{Adjacency and Laplacian.}
Define the adjacency operator $A_{G_n}$ and the degree operator $D_{G_n}$ on
$\ell^2(\mathcal D_n)$ by
\[
(A_{G_n}\psi)(w) := \sum_{w'\sim w} \psi(w'),
\qquad
(D_{G_n}\psi)(w) := \deg(w)\,\psi(w),
\]
where $w'\sim w$ denotes adjacency in $G_n$. The (combinatorial) graph Laplacian is
\[
\Delta_{G_n} := D_{G_n}-A_{G_n},
\]
and the associated diffusion generator is
\[
L_{G_n} := -\Delta_{G_n}.
\]
Then $\Delta_{G_n}$ is self-adjoint and positive semidefinite, while $L_{G_n}$ is
self-adjoint and negative semidefinite.

\paragraph{Discrete heat and Schr\"odinger equations.}
The heat equation on the tier graph is the linear ODE
\[
\partial_\tau u = L_{G_n}u,
\]
with solution $u(\tau)=e^{\tau L_{G_n}}u(0)$. Because $-\Delta_{G_n}$ is
self-adjoint and nonpositive, $e^{\tau L_{G_n}}$ is a contraction semigroup.
The corresponding unitary ``free'' Schr\"odinger evolution is
\[
i\,\partial_t \psi = -L_{G_n}\psi = \Delta_{G_n}\psi,
\]
with solution $\psi(t)=e^{-it\Delta_{G_n}}\psi(0)$.

\begin{remark}
This optional tier-graph framework does not fix a preferred choice of adjacency
$G_n$ and is not used elsewhere in the paper. Its purpose is to make explicit
that, once a tier-local notion of neighbourhood is specified, discrete diffusion
and Schr\"odinger-type evolutions on the Catalan state space follow by standard
graph-Laplacian constructions (compare Remark~\ref{rem:dyck-conditioned-drift-scaling}
and Remark~\ref{rem:dyck-conditioned-kernel-doob} for the tier-growth Markov
structure induced by Dyck conditioning).
\end{remark}

% ---------------------------------------------------------------------------
% Bibitems used only in this supplement (kept here for copy/paste convenience).
% If you re-`\input{supplemental-operators.tex}` into the main paper, these
% entries should live in the main `thebibliography` environment.
% ---------------------------------------------------------------------------
\iffalse
\begin{thebibliography}{99}

\bibitem{abhy2018scatteringforms}
N.~Arkani-Hamed, Y.~Bai, S.~He, and G.~Yan.
\newblock Scattering forms and the positive geometry of kinematics, color and
the worldsheet.
\newblock {\em JHEP} \textbf{05} (2018) 096.
\newblock arXiv:1711.09102.

\bibitem{banerjee2018stokes}
P.~Banerjee, A.~Laddha, and P.~Raman.
\newblock Stokes polytopes: The positive geometry for $\phi^4$ interactions.
\newblock arXiv:1811.05904, 2018.

\bibitem{herrmann2022positivegeometry}
E.~Herrmann and J.~Trnka.
\newblock Positive geometry of scattering amplitudes.
\newblock arXiv:2203.13018, 2022.

\end{thebibliography}
\fi
` into the main paper, these
% entries should live in the main `thebibliography` environment.
% ---------------------------------------------------------------------------
\iffalse
\begin{thebibliography}{99}

\bibitem{abhy2018scatteringforms}
N.~Arkani-Hamed, Y.~Bai, S.~He, and G.~Yan.
\newblock Scattering forms and the positive geometry of kinematics, color and
the worldsheet.
\newblock {\em JHEP} \textbf{05} (2018) 096.
\newblock arXiv:1711.09102.

\bibitem{banerjee2018stokes}
P.~Banerjee, A.~Laddha, and P.~Raman.
\newblock Stokes polytopes: The positive geometry for $\phi^4$ interactions.
\newblock arXiv:1811.05904, 2018.

\bibitem{herrmann2022positivegeometry}
E.~Herrmann and J.~Trnka.
\newblock Positive geometry of scattering amplitudes.
\newblock arXiv:2203.13018, 2022.

\end{thebibliography}
\fi
` at the former location in
% Appendix "Additional Technical Notes".

\subsection{Fields on words, prefixes, and nodes (optional)}
\label{subsec:fields}

We use the word ``field'' as shorthand for a complex-valued function on one of
the Catalan objects already in play. Several closely related state spaces are
useful in different contexts.

\paragraph{Fields on completed histories (fixed tier).}
Fix $n$ and consider a function $\Phi_n:\mathcal D_n\to\mathbb{C}$ assigning an
amplitude (or observable value) to each completed history $w\in\mathcal D_n$.
The associated Hilbert space is $\ell^2(\mathcal D_n)$ with inner product
\[
\langle \psi,\phi\rangle := \sum_{w\in\mathcal D_n} \overline{\psi(w)}\,\phi(w).
\]

\paragraph{Fields on prefixes (the full cone).}
Let $\mathcal{C}$ denote the set of Dyck prefixes (admissible partial histories).
A prefix field is a function $\Phi:\mathcal{C}\to\mathbb{C}$, which may be
restricted to a fixed length slice
$\mathcal{C}^{(k)}:=\{p\in\mathcal{C}: |p|=k\}$ when needed.

\paragraph{Fields on nodes of a fixed tree.}
Given $w\in\mathcal D_n$, let $T(w)$ be its associated full binary tree. A
node field is a function $\phi_w:\mathrm{Int}(T(w))\to\mathbb{C}$ on the internal
nodes of that tree.

\begin{remark}
These notions live on different objects (tiers, the prefix poset, or a single
tree) and are independent of any within-tier ordering convention on
$\mathcal D_n$.
\end{remark}

\subsection{Subtree indicators as a multiscale spanning family (optional)}
\label{subsec:subtree-indicators}

Let $T$ be a finite rooted tree and write $\mathrm{Int}(T)$ for its internal
nodes. Each $v\in\mathrm{Int}(T)$ determines a rooted subtree $T_v$, and hence a
subset $\mathrm{Int}(T_v)\subseteq \mathrm{Int}(T)$. Define the subtree indicator
\[
\chi_v:\mathrm{Int}(T)\to\{0,1\},
\qquad
\chi_v(u):=\mathbf{1}\{u\in \mathrm{Int}(T_v)\}.
\]

\begin{lemma}[Subtree indicators form a basis]
\label{lem:subtree-indicator-basis}
The family $\{\chi_v: v\in \mathrm{Int}(T)\}$ is a basis of the vector space of
complex-valued functions on $\mathrm{Int}(T)$.
\end{lemma}

\begin{proof}
Order the internal nodes by nonincreasing depth (deepest first), and let $M$ be
the square matrix with entries $M_{uv}:=\chi_v(u)$. Then $M_{vv}=1$ for all $v$,
while $M_{uv}=0$ whenever $u$ precedes $v$ in this order (a node cannot be a
descendant of a deeper node). Thus $M$ is triangular with ones on the diagonal,
hence invertible. Therefore the indicators are linearly independent and, since
their number equals $\#\mathrm{Int}(T)$, they form a basis.
\end{proof}

\begin{corollary}[Explicit inversion]
\label{cor:subtree-indicator-inversion}
Let $f:\mathrm{Int}(T)\to\mathbb{C}$ be any function. There is a unique family
of coefficients $\{a_v\}_{v\in\mathrm{Int}(T)}$ such that
\[
f \;=\; \sum_{v\in\mathrm{Int}(T)} a_v\,\chi_v.
\]
Writing $\mathrm{par}(v)$ for the parent of $v$ (for $v\neq \mathrm{root}(T)$),
these coefficients are given by
\[
a_{\mathrm{root}(T)} = f(\mathrm{root}(T)),
\qquad
a_v = f(v)-f(\mathrm{par}(v)) \quad (v\neq \mathrm{root}(T)).
\]
\end{corollary}

\begin{proof}
For each $u\in\mathrm{Int}(T)$,
$(\sum_v a_v\chi_v)(u)=\sum_{v:\,u\in\mathrm{Int}(T_v)} a_v
=\sum_{v\preceq u} a_v$, where $v\preceq u$ means that $v$ is an ancestor of
$u$. With the stated choice of coefficients, this ancestor sum telescopes along
the unique root-to-$u$ chain to yield $f(u)$. Uniqueness follows from
Lemma~\ref{lem:subtree-indicator-basis}.
\end{proof}

\begin{remark}
This basis is ``multiscale'': indicators of deep subtrees localize to fine
regions of $T$, while indicators near the root encode coarse structure. Any
choice of orthonormalization yields an orthonormal basis adapted to the rooted
tree geometry.
\end{remark}

\subsection{Operators on a fixed history tree (optional)}
\label{subsec:tree-operators}

In addition to tier-wise state spaces (fields on $\mathcal D_n$), one may also
consider dynamics \emph{within} a fixed realized history by placing operators on
the internal nodes of its tree.

\paragraph{Node Hilbert space.}
Fix $w\in\mathcal D_n$ and let $T(w)$ be its associated full binary tree. Write
$V_w:=\mathrm{Int}(T(w))$ and consider $\ell^2(V_w)$ with inner product
$\langle \psi,\phi\rangle := \sum_{v\in V_w}\overline{\psi(v)}\,\phi(v)$.

\paragraph{Adjacency and Laplacian.}
Let $G_w=(V_w,E_w)$ be any finite undirected graph on $V_w$ (for example, connect
each internal node to its internal children). Define $A_{G_w}$, $D_{G_w}$, and
the graph Laplacian and generator by
\[
\Delta_{G_w}:=D_{G_w}-A_{G_w},
\qquad
L_{G_w}:=-\Delta_{G_w}.
\]

\paragraph{Heat and Schr\"odinger evolutions.}
The corresponding ``internal-time'' heat equation is
\[
\partial_\tau u = L_{G_w}u,
\]
and the corresponding unitary Schr\"odinger evolution is
\[
i\,\partial_t \psi = -L_{G_w}\psi = \Delta_{G_w}\psi.
\]

\begin{remark}
This within-history operator framework is independent of the tier-growth Markov
dynamics and of coherent summation over histories: it simply records that, once
a graph structure is specified on the internal nodes of a fixed Catalan tree,
standard graph-Laplacian constructions yield discrete diffusion and
Schr\"odinger-type evolutions on that fixed combinatorial background.
\end{remark}

\subsection{Operators on tier slices (optional)}
\label{subsec:tier-operators}

The main text emphasizes two dynamics on the Catalan substrate: tier growth
(prefix extension) and coherent summation over histories. Independently, one may
also consider \emph{slice dynamics} on a fixed tier by endowing the finite set
$\mathcal D_n$ with an auxiliary adjacency graph. This subsection records the
standard operator framework for such constructions.

\paragraph{Tier Hilbert space.}
We take the tier state space to be $\ell^2(\mathcal D_n)$ as in
Section~\ref{subsec:fields}.

\paragraph{Adjacency graphs.}
Let $G_n=(\mathcal D_n,E_n)$ be any finite undirected graph on $\mathcal D_n$.
The choice of $G_n$ is additional structure: different graphs induce different
notions of locality on the tier. A canonical example is the rotation graph (the
associahedron adjacency) on full binary trees, where edges correspond to single
associativity rotations \cite{stanley-catalan,cheneviere2022linear}.

\paragraph{Associahedra and planar tree amplitudes (scattering-amplitude tie-in).}
The associahedron adjacency on $\mathcal D_n$ is also natural from the
perspective of scattering amplitudes. For the planar tree-level sector of
bi-adjoint cubic scalar theory ($\phi^3$), Arkani-Hamed, Bai, He, and Yan
identify an associahedron in planar kinematic space and show that the tree
amplitude is the corresponding canonical form of this positive geometry
\cite{abhy2018scatteringforms}. From this viewpoint, the Catalan enumeration of
planar cubic tree diagrams is not merely counting: the associahedron organizes
factorization channels geometrically, and different triangulations correspond to
different diagrammatic expansions of the same canonical form (see, e.g., the
review \cite{herrmann2022positivegeometry}). For quartic interactions, an
analogous positive-geometry description involves Stokes polytopes rather than
associahedra \cite{banerjee2018stokes}.

\paragraph{Tamari/Dyck/alt-Tamari choices on the same tier.}
The point of introducing an auxiliary graph $G_n$ is that the underlying state
set $\mathcal D_n$ supports multiple natural notions of tier-locality coming
from classical Catalan posets. The rotation graph is the undirected adjacency
underlying the Tamari order; one may likewise equip $\mathcal D_n$ with
adjacency induced by the Dyck (``Stanley'') lattice on Dyck paths, or more
generally by the family of $\delta$-Tamari (alt-Tamari) posets interpolating
between these extremes. These alternatives use different covering relations on
the same Catalan tier and therefore induce different graph Laplacians
$\Delta_{G_n}$ and different ``free'' tier Hamiltonians, but they live on a
common configuration space $\mathcal D_n$ \cite{stanley-catalan,cheneviere2022linear}.

\paragraph{Linear intervals as ``1D corridors'' in Catalan posets.}
A useful robustness fact is that certain one-dimensional substructures are
invariant across these Catalan posets: Chenevi\`ere proves that, for each fixed
$n$ and each height parameter $k$ (in the sense of \cite{cheneviere2022linear}),
the Tamari lattice and the Dyck lattice have the same number of \emph{linear
intervals} (intervals whose Hasse diagram is a chain), and moreover all
alt-Tamari posets share this same count at each height
\cite{cheneviere2022linear}. In the present language, this says that the number
of ``diamond-free corridors'' (regions with a unique maximal chain) is stable
under a wide class of tier-local adjacency choices, reinforcing the theme that
many distinct dynamics can be layered on a single Catalan substrate without
changing its most rigid combinatorial invariants.

\paragraph{Adjacency and Laplacian.}
Define the adjacency operator $A_{G_n}$ and the degree operator $D_{G_n}$ on
$\ell^2(\mathcal D_n)$ by
\[
(A_{G_n}\psi)(w) := \sum_{w'\sim w} \psi(w'),
\qquad
(D_{G_n}\psi)(w) := \deg(w)\,\psi(w),
\]
where $w'\sim w$ denotes adjacency in $G_n$. The (combinatorial) graph Laplacian is
\[
\Delta_{G_n} := D_{G_n}-A_{G_n},
\]
and the associated diffusion generator is
\[
L_{G_n} := -\Delta_{G_n}.
\]
Then $\Delta_{G_n}$ is self-adjoint and positive semidefinite, while $L_{G_n}$ is
self-adjoint and negative semidefinite.

\paragraph{Discrete heat and Schr\"odinger equations.}
The heat equation on the tier graph is the linear ODE
\[
\partial_\tau u = L_{G_n}u,
\]
with solution $u(\tau)=e^{\tau L_{G_n}}u(0)$. Because $-\Delta_{G_n}$ is
self-adjoint and nonpositive, $e^{\tau L_{G_n}}$ is a contraction semigroup.
The corresponding unitary ``free'' Schr\"odinger evolution is
\[
i\,\partial_t \psi = -L_{G_n}\psi = \Delta_{G_n}\psi,
\]
with solution $\psi(t)=e^{-it\Delta_{G_n}}\psi(0)$.

\begin{remark}
This optional tier-graph framework does not fix a preferred choice of adjacency
$G_n$ and is not used elsewhere in the paper. Its purpose is to make explicit
that, once a tier-local notion of neighbourhood is specified, discrete diffusion
and Schr\"odinger-type evolutions on the Catalan state space follow by standard
graph-Laplacian constructions (compare Remark~\ref{rem:dyck-conditioned-drift-scaling}
and Remark~\ref{rem:dyck-conditioned-kernel-doob} for the tier-growth Markov
structure induced by Dyck conditioning).
\end{remark}

% ---------------------------------------------------------------------------
% Bibitems used only in this supplement (kept here for copy/paste convenience).
% If you re-`% Supplemental material extracted from `docs/catalan-light-cone.tex`.
% This block was removed from the compiled arXiv PDF to keep the main v1 lean.
% Intended use: `\% Supplemental material extracted from `docs/catalan-light-cone.tex`.
% This block was removed from the compiled arXiv PDF to keep the main v1 lean.
% Intended use: `\\input{supplemental-operators.tex}` at the former location in
% Appendix "Additional Technical Notes".

\subsection{Fields on words, prefixes, and nodes (optional)}
\label{subsec:fields}

We use the word ``field'' as shorthand for a complex-valued function on one of
the Catalan objects already in play. Several closely related state spaces are
useful in different contexts.

\paragraph{Fields on completed histories (fixed tier).}
Fix $n$ and consider a function $\Phi_n:\mathcal D_n\to\mathbb{C}$ assigning an
amplitude (or observable value) to each completed history $w\in\mathcal D_n$.
The associated Hilbert space is $\ell^2(\mathcal D_n)$ with inner product
\[
\langle \psi,\phi\rangle := \sum_{w\in\mathcal D_n} \overline{\psi(w)}\,\phi(w).
\]

\paragraph{Fields on prefixes (the full cone).}
Let $\mathcal{C}$ denote the set of Dyck prefixes (admissible partial histories).
A prefix field is a function $\Phi:\mathcal{C}\to\mathbb{C}$, which may be
restricted to a fixed length slice
$\mathcal{C}^{(k)}:=\{p\in\mathcal{C}: |p|=k\}$ when needed.

\paragraph{Fields on nodes of a fixed tree.}
Given $w\in\mathcal D_n$, let $T(w)$ be its associated full binary tree. A
node field is a function $\phi_w:\mathrm{Int}(T(w))\to\mathbb{C}$ on the internal
nodes of that tree.

\begin{remark}
These notions live on different objects (tiers, the prefix poset, or a single
tree) and are independent of any within-tier ordering convention on
$\mathcal D_n$.
\end{remark}

\subsection{Subtree indicators as a multiscale spanning family (optional)}
\label{subsec:subtree-indicators}

Let $T$ be a finite rooted tree and write $\mathrm{Int}(T)$ for its internal
nodes. Each $v\in\mathrm{Int}(T)$ determines a rooted subtree $T_v$, and hence a
subset $\mathrm{Int}(T_v)\subseteq \mathrm{Int}(T)$. Define the subtree indicator
\[
\chi_v:\mathrm{Int}(T)\to\{0,1\},
\qquad
\chi_v(u):=\mathbf{1}\{u\in \mathrm{Int}(T_v)\}.
\]

\begin{lemma}[Subtree indicators form a basis]
\label{lem:subtree-indicator-basis}
The family $\{\chi_v: v\in \mathrm{Int}(T)\}$ is a basis of the vector space of
complex-valued functions on $\mathrm{Int}(T)$.
\end{lemma}

\begin{proof}
Order the internal nodes by nonincreasing depth (deepest first), and let $M$ be
the square matrix with entries $M_{uv}:=\chi_v(u)$. Then $M_{vv}=1$ for all $v$,
while $M_{uv}=0$ whenever $u$ precedes $v$ in this order (a node cannot be a
descendant of a deeper node). Thus $M$ is triangular with ones on the diagonal,
hence invertible. Therefore the indicators are linearly independent and, since
their number equals $\#\mathrm{Int}(T)$, they form a basis.
\end{proof}

\begin{corollary}[Explicit inversion]
\label{cor:subtree-indicator-inversion}
Let $f:\mathrm{Int}(T)\to\mathbb{C}$ be any function. There is a unique family
of coefficients $\{a_v\}_{v\in\mathrm{Int}(T)}$ such that
\[
f \;=\; \sum_{v\in\mathrm{Int}(T)} a_v\,\chi_v.
\]
Writing $\mathrm{par}(v)$ for the parent of $v$ (for $v\neq \mathrm{root}(T)$),
these coefficients are given by
\[
a_{\mathrm{root}(T)} = f(\mathrm{root}(T)),
\qquad
a_v = f(v)-f(\mathrm{par}(v)) \quad (v\neq \mathrm{root}(T)).
\]
\end{corollary}

\begin{proof}
For each $u\in\mathrm{Int}(T)$,
$(\sum_v a_v\chi_v)(u)=\sum_{v:\,u\in\mathrm{Int}(T_v)} a_v
=\sum_{v\preceq u} a_v$, where $v\preceq u$ means that $v$ is an ancestor of
$u$. With the stated choice of coefficients, this ancestor sum telescopes along
the unique root-to-$u$ chain to yield $f(u)$. Uniqueness follows from
Lemma~\ref{lem:subtree-indicator-basis}.
\end{proof}

\begin{remark}
This basis is ``multiscale'': indicators of deep subtrees localize to fine
regions of $T$, while indicators near the root encode coarse structure. Any
choice of orthonormalization yields an orthonormal basis adapted to the rooted
tree geometry.
\end{remark}

\subsection{Operators on a fixed history tree (optional)}
\label{subsec:tree-operators}

In addition to tier-wise state spaces (fields on $\mathcal D_n$), one may also
consider dynamics \emph{within} a fixed realized history by placing operators on
the internal nodes of its tree.

\paragraph{Node Hilbert space.}
Fix $w\in\mathcal D_n$ and let $T(w)$ be its associated full binary tree. Write
$V_w:=\mathrm{Int}(T(w))$ and consider $\ell^2(V_w)$ with inner product
$\langle \psi,\phi\rangle := \sum_{v\in V_w}\overline{\psi(v)}\,\phi(v)$.

\paragraph{Adjacency and Laplacian.}
Let $G_w=(V_w,E_w)$ be any finite undirected graph on $V_w$ (for example, connect
each internal node to its internal children). Define $A_{G_w}$, $D_{G_w}$, and
the graph Laplacian and generator by
\[
\Delta_{G_w}:=D_{G_w}-A_{G_w},
\qquad
L_{G_w}:=-\Delta_{G_w}.
\]

\paragraph{Heat and Schr\"odinger evolutions.}
The corresponding ``internal-time'' heat equation is
\[
\partial_\tau u = L_{G_w}u,
\]
and the corresponding unitary Schr\"odinger evolution is
\[
i\,\partial_t \psi = -L_{G_w}\psi = \Delta_{G_w}\psi.
\]

\begin{remark}
This within-history operator framework is independent of the tier-growth Markov
dynamics and of coherent summation over histories: it simply records that, once
a graph structure is specified on the internal nodes of a fixed Catalan tree,
standard graph-Laplacian constructions yield discrete diffusion and
Schr\"odinger-type evolutions on that fixed combinatorial background.
\end{remark}

\subsection{Operators on tier slices (optional)}
\label{subsec:tier-operators}

The main text emphasizes two dynamics on the Catalan substrate: tier growth
(prefix extension) and coherent summation over histories. Independently, one may
also consider \emph{slice dynamics} on a fixed tier by endowing the finite set
$\mathcal D_n$ with an auxiliary adjacency graph. This subsection records the
standard operator framework for such constructions.

\paragraph{Tier Hilbert space.}
We take the tier state space to be $\ell^2(\mathcal D_n)$ as in
Section~\ref{subsec:fields}.

\paragraph{Adjacency graphs.}
Let $G_n=(\mathcal D_n,E_n)$ be any finite undirected graph on $\mathcal D_n$.
The choice of $G_n$ is additional structure: different graphs induce different
notions of locality on the tier. A canonical example is the rotation graph (the
associahedron adjacency) on full binary trees, where edges correspond to single
associativity rotations \cite{stanley-catalan,cheneviere2022linear}.

\paragraph{Associahedra and planar tree amplitudes (scattering-amplitude tie-in).}
The associahedron adjacency on $\mathcal D_n$ is also natural from the
perspective of scattering amplitudes. For the planar tree-level sector of
bi-adjoint cubic scalar theory ($\phi^3$), Arkani-Hamed, Bai, He, and Yan
identify an associahedron in planar kinematic space and show that the tree
amplitude is the corresponding canonical form of this positive geometry
\cite{abhy2018scatteringforms}. From this viewpoint, the Catalan enumeration of
planar cubic tree diagrams is not merely counting: the associahedron organizes
factorization channels geometrically, and different triangulations correspond to
different diagrammatic expansions of the same canonical form (see, e.g., the
review \cite{herrmann2022positivegeometry}). For quartic interactions, an
analogous positive-geometry description involves Stokes polytopes rather than
associahedra \cite{banerjee2018stokes}.

\paragraph{Tamari/Dyck/alt-Tamari choices on the same tier.}
The point of introducing an auxiliary graph $G_n$ is that the underlying state
set $\mathcal D_n$ supports multiple natural notions of tier-locality coming
from classical Catalan posets. The rotation graph is the undirected adjacency
underlying the Tamari order; one may likewise equip $\mathcal D_n$ with
adjacency induced by the Dyck (``Stanley'') lattice on Dyck paths, or more
generally by the family of $\delta$-Tamari (alt-Tamari) posets interpolating
between these extremes. These alternatives use different covering relations on
the same Catalan tier and therefore induce different graph Laplacians
$\Delta_{G_n}$ and different ``free'' tier Hamiltonians, but they live on a
common configuration space $\mathcal D_n$ \cite{stanley-catalan,cheneviere2022linear}.

\paragraph{Linear intervals as ``1D corridors'' in Catalan posets.}
A useful robustness fact is that certain one-dimensional substructures are
invariant across these Catalan posets: Chenevi\`ere proves that, for each fixed
$n$ and each height parameter $k$ (in the sense of \cite{cheneviere2022linear}),
the Tamari lattice and the Dyck lattice have the same number of \emph{linear
intervals} (intervals whose Hasse diagram is a chain), and moreover all
alt-Tamari posets share this same count at each height
\cite{cheneviere2022linear}. In the present language, this says that the number
of ``diamond-free corridors'' (regions with a unique maximal chain) is stable
under a wide class of tier-local adjacency choices, reinforcing the theme that
many distinct dynamics can be layered on a single Catalan substrate without
changing its most rigid combinatorial invariants.

\paragraph{Adjacency and Laplacian.}
Define the adjacency operator $A_{G_n}$ and the degree operator $D_{G_n}$ on
$\ell^2(\mathcal D_n)$ by
\[
(A_{G_n}\psi)(w) := \sum_{w'\sim w} \psi(w'),
\qquad
(D_{G_n}\psi)(w) := \deg(w)\,\psi(w),
\]
where $w'\sim w$ denotes adjacency in $G_n$. The (combinatorial) graph Laplacian is
\[
\Delta_{G_n} := D_{G_n}-A_{G_n},
\]
and the associated diffusion generator is
\[
L_{G_n} := -\Delta_{G_n}.
\]
Then $\Delta_{G_n}$ is self-adjoint and positive semidefinite, while $L_{G_n}$ is
self-adjoint and negative semidefinite.

\paragraph{Discrete heat and Schr\"odinger equations.}
The heat equation on the tier graph is the linear ODE
\[
\partial_\tau u = L_{G_n}u,
\]
with solution $u(\tau)=e^{\tau L_{G_n}}u(0)$. Because $-\Delta_{G_n}$ is
self-adjoint and nonpositive, $e^{\tau L_{G_n}}$ is a contraction semigroup.
The corresponding unitary ``free'' Schr\"odinger evolution is
\[
i\,\partial_t \psi = -L_{G_n}\psi = \Delta_{G_n}\psi,
\]
with solution $\psi(t)=e^{-it\Delta_{G_n}}\psi(0)$.

\begin{remark}
This optional tier-graph framework does not fix a preferred choice of adjacency
$G_n$ and is not used elsewhere in the paper. Its purpose is to make explicit
that, once a tier-local notion of neighbourhood is specified, discrete diffusion
and Schr\"odinger-type evolutions on the Catalan state space follow by standard
graph-Laplacian constructions (compare Remark~\ref{rem:dyck-conditioned-drift-scaling}
and Remark~\ref{rem:dyck-conditioned-kernel-doob} for the tier-growth Markov
structure induced by Dyck conditioning).
\end{remark}

% ---------------------------------------------------------------------------
% Bibitems used only in this supplement (kept here for copy/paste convenience).
% If you re-`\input{supplemental-operators.tex}` into the main paper, these
% entries should live in the main `thebibliography` environment.
% ---------------------------------------------------------------------------
\iffalse
\begin{thebibliography}{99}

\bibitem{abhy2018scatteringforms}
N.~Arkani-Hamed, Y.~Bai, S.~He, and G.~Yan.
\newblock Scattering forms and the positive geometry of kinematics, color and
the worldsheet.
\newblock {\em JHEP} \textbf{05} (2018) 096.
\newblock arXiv:1711.09102.

\bibitem{banerjee2018stokes}
P.~Banerjee, A.~Laddha, and P.~Raman.
\newblock Stokes polytopes: The positive geometry for $\phi^4$ interactions.
\newblock arXiv:1811.05904, 2018.

\bibitem{herrmann2022positivegeometry}
E.~Herrmann and J.~Trnka.
\newblock Positive geometry of scattering amplitudes.
\newblock arXiv:2203.13018, 2022.

\end{thebibliography}
\fi
` at the former location in
% Appendix "Additional Technical Notes".

\subsection{Fields on words, prefixes, and nodes (optional)}
\label{subsec:fields}

We use the word ``field'' as shorthand for a complex-valued function on one of
the Catalan objects already in play. Several closely related state spaces are
useful in different contexts.

\paragraph{Fields on completed histories (fixed tier).}
Fix $n$ and consider a function $\Phi_n:\mathcal D_n\to\mathbb{C}$ assigning an
amplitude (or observable value) to each completed history $w\in\mathcal D_n$.
The associated Hilbert space is $\ell^2(\mathcal D_n)$ with inner product
\[
\langle \psi,\phi\rangle := \sum_{w\in\mathcal D_n} \overline{\psi(w)}\,\phi(w).
\]

\paragraph{Fields on prefixes (the full cone).}
Let $\mathcal{C}$ denote the set of Dyck prefixes (admissible partial histories).
A prefix field is a function $\Phi:\mathcal{C}\to\mathbb{C}$, which may be
restricted to a fixed length slice
$\mathcal{C}^{(k)}:=\{p\in\mathcal{C}: |p|=k\}$ when needed.

\paragraph{Fields on nodes of a fixed tree.}
Given $w\in\mathcal D_n$, let $T(w)$ be its associated full binary tree. A
node field is a function $\phi_w:\mathrm{Int}(T(w))\to\mathbb{C}$ on the internal
nodes of that tree.

\begin{remark}
These notions live on different objects (tiers, the prefix poset, or a single
tree) and are independent of any within-tier ordering convention on
$\mathcal D_n$.
\end{remark}

\subsection{Subtree indicators as a multiscale spanning family (optional)}
\label{subsec:subtree-indicators}

Let $T$ be a finite rooted tree and write $\mathrm{Int}(T)$ for its internal
nodes. Each $v\in\mathrm{Int}(T)$ determines a rooted subtree $T_v$, and hence a
subset $\mathrm{Int}(T_v)\subseteq \mathrm{Int}(T)$. Define the subtree indicator
\[
\chi_v:\mathrm{Int}(T)\to\{0,1\},
\qquad
\chi_v(u):=\mathbf{1}\{u\in \mathrm{Int}(T_v)\}.
\]

\begin{lemma}[Subtree indicators form a basis]
\label{lem:subtree-indicator-basis}
The family $\{\chi_v: v\in \mathrm{Int}(T)\}$ is a basis of the vector space of
complex-valued functions on $\mathrm{Int}(T)$.
\end{lemma}

\begin{proof}
Order the internal nodes by nonincreasing depth (deepest first), and let $M$ be
the square matrix with entries $M_{uv}:=\chi_v(u)$. Then $M_{vv}=1$ for all $v$,
while $M_{uv}=0$ whenever $u$ precedes $v$ in this order (a node cannot be a
descendant of a deeper node). Thus $M$ is triangular with ones on the diagonal,
hence invertible. Therefore the indicators are linearly independent and, since
their number equals $\#\mathrm{Int}(T)$, they form a basis.
\end{proof}

\begin{corollary}[Explicit inversion]
\label{cor:subtree-indicator-inversion}
Let $f:\mathrm{Int}(T)\to\mathbb{C}$ be any function. There is a unique family
of coefficients $\{a_v\}_{v\in\mathrm{Int}(T)}$ such that
\[
f \;=\; \sum_{v\in\mathrm{Int}(T)} a_v\,\chi_v.
\]
Writing $\mathrm{par}(v)$ for the parent of $v$ (for $v\neq \mathrm{root}(T)$),
these coefficients are given by
\[
a_{\mathrm{root}(T)} = f(\mathrm{root}(T)),
\qquad
a_v = f(v)-f(\mathrm{par}(v)) \quad (v\neq \mathrm{root}(T)).
\]
\end{corollary}

\begin{proof}
For each $u\in\mathrm{Int}(T)$,
$(\sum_v a_v\chi_v)(u)=\sum_{v:\,u\in\mathrm{Int}(T_v)} a_v
=\sum_{v\preceq u} a_v$, where $v\preceq u$ means that $v$ is an ancestor of
$u$. With the stated choice of coefficients, this ancestor sum telescopes along
the unique root-to-$u$ chain to yield $f(u)$. Uniqueness follows from
Lemma~\ref{lem:subtree-indicator-basis}.
\end{proof}

\begin{remark}
This basis is ``multiscale'': indicators of deep subtrees localize to fine
regions of $T$, while indicators near the root encode coarse structure. Any
choice of orthonormalization yields an orthonormal basis adapted to the rooted
tree geometry.
\end{remark}

\subsection{Operators on a fixed history tree (optional)}
\label{subsec:tree-operators}

In addition to tier-wise state spaces (fields on $\mathcal D_n$), one may also
consider dynamics \emph{within} a fixed realized history by placing operators on
the internal nodes of its tree.

\paragraph{Node Hilbert space.}
Fix $w\in\mathcal D_n$ and let $T(w)$ be its associated full binary tree. Write
$V_w:=\mathrm{Int}(T(w))$ and consider $\ell^2(V_w)$ with inner product
$\langle \psi,\phi\rangle := \sum_{v\in V_w}\overline{\psi(v)}\,\phi(v)$.

\paragraph{Adjacency and Laplacian.}
Let $G_w=(V_w,E_w)$ be any finite undirected graph on $V_w$ (for example, connect
each internal node to its internal children). Define $A_{G_w}$, $D_{G_w}$, and
the graph Laplacian and generator by
\[
\Delta_{G_w}:=D_{G_w}-A_{G_w},
\qquad
L_{G_w}:=-\Delta_{G_w}.
\]

\paragraph{Heat and Schr\"odinger evolutions.}
The corresponding ``internal-time'' heat equation is
\[
\partial_\tau u = L_{G_w}u,
\]
and the corresponding unitary Schr\"odinger evolution is
\[
i\,\partial_t \psi = -L_{G_w}\psi = \Delta_{G_w}\psi.
\]

\begin{remark}
This within-history operator framework is independent of the tier-growth Markov
dynamics and of coherent summation over histories: it simply records that, once
a graph structure is specified on the internal nodes of a fixed Catalan tree,
standard graph-Laplacian constructions yield discrete diffusion and
Schr\"odinger-type evolutions on that fixed combinatorial background.
\end{remark}

\subsection{Operators on tier slices (optional)}
\label{subsec:tier-operators}

The main text emphasizes two dynamics on the Catalan substrate: tier growth
(prefix extension) and coherent summation over histories. Independently, one may
also consider \emph{slice dynamics} on a fixed tier by endowing the finite set
$\mathcal D_n$ with an auxiliary adjacency graph. This subsection records the
standard operator framework for such constructions.

\paragraph{Tier Hilbert space.}
We take the tier state space to be $\ell^2(\mathcal D_n)$ as in
Section~\ref{subsec:fields}.

\paragraph{Adjacency graphs.}
Let $G_n=(\mathcal D_n,E_n)$ be any finite undirected graph on $\mathcal D_n$.
The choice of $G_n$ is additional structure: different graphs induce different
notions of locality on the tier. A canonical example is the rotation graph (the
associahedron adjacency) on full binary trees, where edges correspond to single
associativity rotations \cite{stanley-catalan,cheneviere2022linear}.

\paragraph{Associahedra and planar tree amplitudes (scattering-amplitude tie-in).}
The associahedron adjacency on $\mathcal D_n$ is also natural from the
perspective of scattering amplitudes. For the planar tree-level sector of
bi-adjoint cubic scalar theory ($\phi^3$), Arkani-Hamed, Bai, He, and Yan
identify an associahedron in planar kinematic space and show that the tree
amplitude is the corresponding canonical form of this positive geometry
\cite{abhy2018scatteringforms}. From this viewpoint, the Catalan enumeration of
planar cubic tree diagrams is not merely counting: the associahedron organizes
factorization channels geometrically, and different triangulations correspond to
different diagrammatic expansions of the same canonical form (see, e.g., the
review \cite{herrmann2022positivegeometry}). For quartic interactions, an
analogous positive-geometry description involves Stokes polytopes rather than
associahedra \cite{banerjee2018stokes}.

\paragraph{Tamari/Dyck/alt-Tamari choices on the same tier.}
The point of introducing an auxiliary graph $G_n$ is that the underlying state
set $\mathcal D_n$ supports multiple natural notions of tier-locality coming
from classical Catalan posets. The rotation graph is the undirected adjacency
underlying the Tamari order; one may likewise equip $\mathcal D_n$ with
adjacency induced by the Dyck (``Stanley'') lattice on Dyck paths, or more
generally by the family of $\delta$-Tamari (alt-Tamari) posets interpolating
between these extremes. These alternatives use different covering relations on
the same Catalan tier and therefore induce different graph Laplacians
$\Delta_{G_n}$ and different ``free'' tier Hamiltonians, but they live on a
common configuration space $\mathcal D_n$ \cite{stanley-catalan,cheneviere2022linear}.

\paragraph{Linear intervals as ``1D corridors'' in Catalan posets.}
A useful robustness fact is that certain one-dimensional substructures are
invariant across these Catalan posets: Chenevi\`ere proves that, for each fixed
$n$ and each height parameter $k$ (in the sense of \cite{cheneviere2022linear}),
the Tamari lattice and the Dyck lattice have the same number of \emph{linear
intervals} (intervals whose Hasse diagram is a chain), and moreover all
alt-Tamari posets share this same count at each height
\cite{cheneviere2022linear}. In the present language, this says that the number
of ``diamond-free corridors'' (regions with a unique maximal chain) is stable
under a wide class of tier-local adjacency choices, reinforcing the theme that
many distinct dynamics can be layered on a single Catalan substrate without
changing its most rigid combinatorial invariants.

\paragraph{Adjacency and Laplacian.}
Define the adjacency operator $A_{G_n}$ and the degree operator $D_{G_n}$ on
$\ell^2(\mathcal D_n)$ by
\[
(A_{G_n}\psi)(w) := \sum_{w'\sim w} \psi(w'),
\qquad
(D_{G_n}\psi)(w) := \deg(w)\,\psi(w),
\]
where $w'\sim w$ denotes adjacency in $G_n$. The (combinatorial) graph Laplacian is
\[
\Delta_{G_n} := D_{G_n}-A_{G_n},
\]
and the associated diffusion generator is
\[
L_{G_n} := -\Delta_{G_n}.
\]
Then $\Delta_{G_n}$ is self-adjoint and positive semidefinite, while $L_{G_n}$ is
self-adjoint and negative semidefinite.

\paragraph{Discrete heat and Schr\"odinger equations.}
The heat equation on the tier graph is the linear ODE
\[
\partial_\tau u = L_{G_n}u,
\]
with solution $u(\tau)=e^{\tau L_{G_n}}u(0)$. Because $-\Delta_{G_n}$ is
self-adjoint and nonpositive, $e^{\tau L_{G_n}}$ is a contraction semigroup.
The corresponding unitary ``free'' Schr\"odinger evolution is
\[
i\,\partial_t \psi = -L_{G_n}\psi = \Delta_{G_n}\psi,
\]
with solution $\psi(t)=e^{-it\Delta_{G_n}}\psi(0)$.

\begin{remark}
This optional tier-graph framework does not fix a preferred choice of adjacency
$G_n$ and is not used elsewhere in the paper. Its purpose is to make explicit
that, once a tier-local notion of neighbourhood is specified, discrete diffusion
and Schr\"odinger-type evolutions on the Catalan state space follow by standard
graph-Laplacian constructions (compare Remark~\ref{rem:dyck-conditioned-drift-scaling}
and Remark~\ref{rem:dyck-conditioned-kernel-doob} for the tier-growth Markov
structure induced by Dyck conditioning).
\end{remark}

% ---------------------------------------------------------------------------
% Bibitems used only in this supplement (kept here for copy/paste convenience).
% If you re-`% Supplemental material extracted from `docs/catalan-light-cone.tex`.
% This block was removed from the compiled arXiv PDF to keep the main v1 lean.
% Intended use: `\\input{supplemental-operators.tex}` at the former location in
% Appendix "Additional Technical Notes".

\subsection{Fields on words, prefixes, and nodes (optional)}
\label{subsec:fields}

We use the word ``field'' as shorthand for a complex-valued function on one of
the Catalan objects already in play. Several closely related state spaces are
useful in different contexts.

\paragraph{Fields on completed histories (fixed tier).}
Fix $n$ and consider a function $\Phi_n:\mathcal D_n\to\mathbb{C}$ assigning an
amplitude (or observable value) to each completed history $w\in\mathcal D_n$.
The associated Hilbert space is $\ell^2(\mathcal D_n)$ with inner product
\[
\langle \psi,\phi\rangle := \sum_{w\in\mathcal D_n} \overline{\psi(w)}\,\phi(w).
\]

\paragraph{Fields on prefixes (the full cone).}
Let $\mathcal{C}$ denote the set of Dyck prefixes (admissible partial histories).
A prefix field is a function $\Phi:\mathcal{C}\to\mathbb{C}$, which may be
restricted to a fixed length slice
$\mathcal{C}^{(k)}:=\{p\in\mathcal{C}: |p|=k\}$ when needed.

\paragraph{Fields on nodes of a fixed tree.}
Given $w\in\mathcal D_n$, let $T(w)$ be its associated full binary tree. A
node field is a function $\phi_w:\mathrm{Int}(T(w))\to\mathbb{C}$ on the internal
nodes of that tree.

\begin{remark}
These notions live on different objects (tiers, the prefix poset, or a single
tree) and are independent of any within-tier ordering convention on
$\mathcal D_n$.
\end{remark}

\subsection{Subtree indicators as a multiscale spanning family (optional)}
\label{subsec:subtree-indicators}

Let $T$ be a finite rooted tree and write $\mathrm{Int}(T)$ for its internal
nodes. Each $v\in\mathrm{Int}(T)$ determines a rooted subtree $T_v$, and hence a
subset $\mathrm{Int}(T_v)\subseteq \mathrm{Int}(T)$. Define the subtree indicator
\[
\chi_v:\mathrm{Int}(T)\to\{0,1\},
\qquad
\chi_v(u):=\mathbf{1}\{u\in \mathrm{Int}(T_v)\}.
\]

\begin{lemma}[Subtree indicators form a basis]
\label{lem:subtree-indicator-basis}
The family $\{\chi_v: v\in \mathrm{Int}(T)\}$ is a basis of the vector space of
complex-valued functions on $\mathrm{Int}(T)$.
\end{lemma}

\begin{proof}
Order the internal nodes by nonincreasing depth (deepest first), and let $M$ be
the square matrix with entries $M_{uv}:=\chi_v(u)$. Then $M_{vv}=1$ for all $v$,
while $M_{uv}=0$ whenever $u$ precedes $v$ in this order (a node cannot be a
descendant of a deeper node). Thus $M$ is triangular with ones on the diagonal,
hence invertible. Therefore the indicators are linearly independent and, since
their number equals $\#\mathrm{Int}(T)$, they form a basis.
\end{proof}

\begin{corollary}[Explicit inversion]
\label{cor:subtree-indicator-inversion}
Let $f:\mathrm{Int}(T)\to\mathbb{C}$ be any function. There is a unique family
of coefficients $\{a_v\}_{v\in\mathrm{Int}(T)}$ such that
\[
f \;=\; \sum_{v\in\mathrm{Int}(T)} a_v\,\chi_v.
\]
Writing $\mathrm{par}(v)$ for the parent of $v$ (for $v\neq \mathrm{root}(T)$),
these coefficients are given by
\[
a_{\mathrm{root}(T)} = f(\mathrm{root}(T)),
\qquad
a_v = f(v)-f(\mathrm{par}(v)) \quad (v\neq \mathrm{root}(T)).
\]
\end{corollary}

\begin{proof}
For each $u\in\mathrm{Int}(T)$,
$(\sum_v a_v\chi_v)(u)=\sum_{v:\,u\in\mathrm{Int}(T_v)} a_v
=\sum_{v\preceq u} a_v$, where $v\preceq u$ means that $v$ is an ancestor of
$u$. With the stated choice of coefficients, this ancestor sum telescopes along
the unique root-to-$u$ chain to yield $f(u)$. Uniqueness follows from
Lemma~\ref{lem:subtree-indicator-basis}.
\end{proof}

\begin{remark}
This basis is ``multiscale'': indicators of deep subtrees localize to fine
regions of $T$, while indicators near the root encode coarse structure. Any
choice of orthonormalization yields an orthonormal basis adapted to the rooted
tree geometry.
\end{remark}

\subsection{Operators on a fixed history tree (optional)}
\label{subsec:tree-operators}

In addition to tier-wise state spaces (fields on $\mathcal D_n$), one may also
consider dynamics \emph{within} a fixed realized history by placing operators on
the internal nodes of its tree.

\paragraph{Node Hilbert space.}
Fix $w\in\mathcal D_n$ and let $T(w)$ be its associated full binary tree. Write
$V_w:=\mathrm{Int}(T(w))$ and consider $\ell^2(V_w)$ with inner product
$\langle \psi,\phi\rangle := \sum_{v\in V_w}\overline{\psi(v)}\,\phi(v)$.

\paragraph{Adjacency and Laplacian.}
Let $G_w=(V_w,E_w)$ be any finite undirected graph on $V_w$ (for example, connect
each internal node to its internal children). Define $A_{G_w}$, $D_{G_w}$, and
the graph Laplacian and generator by
\[
\Delta_{G_w}:=D_{G_w}-A_{G_w},
\qquad
L_{G_w}:=-\Delta_{G_w}.
\]

\paragraph{Heat and Schr\"odinger evolutions.}
The corresponding ``internal-time'' heat equation is
\[
\partial_\tau u = L_{G_w}u,
\]
and the corresponding unitary Schr\"odinger evolution is
\[
i\,\partial_t \psi = -L_{G_w}\psi = \Delta_{G_w}\psi.
\]

\begin{remark}
This within-history operator framework is independent of the tier-growth Markov
dynamics and of coherent summation over histories: it simply records that, once
a graph structure is specified on the internal nodes of a fixed Catalan tree,
standard graph-Laplacian constructions yield discrete diffusion and
Schr\"odinger-type evolutions on that fixed combinatorial background.
\end{remark}

\subsection{Operators on tier slices (optional)}
\label{subsec:tier-operators}

The main text emphasizes two dynamics on the Catalan substrate: tier growth
(prefix extension) and coherent summation over histories. Independently, one may
also consider \emph{slice dynamics} on a fixed tier by endowing the finite set
$\mathcal D_n$ with an auxiliary adjacency graph. This subsection records the
standard operator framework for such constructions.

\paragraph{Tier Hilbert space.}
We take the tier state space to be $\ell^2(\mathcal D_n)$ as in
Section~\ref{subsec:fields}.

\paragraph{Adjacency graphs.}
Let $G_n=(\mathcal D_n,E_n)$ be any finite undirected graph on $\mathcal D_n$.
The choice of $G_n$ is additional structure: different graphs induce different
notions of locality on the tier. A canonical example is the rotation graph (the
associahedron adjacency) on full binary trees, where edges correspond to single
associativity rotations \cite{stanley-catalan,cheneviere2022linear}.

\paragraph{Associahedra and planar tree amplitudes (scattering-amplitude tie-in).}
The associahedron adjacency on $\mathcal D_n$ is also natural from the
perspective of scattering amplitudes. For the planar tree-level sector of
bi-adjoint cubic scalar theory ($\phi^3$), Arkani-Hamed, Bai, He, and Yan
identify an associahedron in planar kinematic space and show that the tree
amplitude is the corresponding canonical form of this positive geometry
\cite{abhy2018scatteringforms}. From this viewpoint, the Catalan enumeration of
planar cubic tree diagrams is not merely counting: the associahedron organizes
factorization channels geometrically, and different triangulations correspond to
different diagrammatic expansions of the same canonical form (see, e.g., the
review \cite{herrmann2022positivegeometry}). For quartic interactions, an
analogous positive-geometry description involves Stokes polytopes rather than
associahedra \cite{banerjee2018stokes}.

\paragraph{Tamari/Dyck/alt-Tamari choices on the same tier.}
The point of introducing an auxiliary graph $G_n$ is that the underlying state
set $\mathcal D_n$ supports multiple natural notions of tier-locality coming
from classical Catalan posets. The rotation graph is the undirected adjacency
underlying the Tamari order; one may likewise equip $\mathcal D_n$ with
adjacency induced by the Dyck (``Stanley'') lattice on Dyck paths, or more
generally by the family of $\delta$-Tamari (alt-Tamari) posets interpolating
between these extremes. These alternatives use different covering relations on
the same Catalan tier and therefore induce different graph Laplacians
$\Delta_{G_n}$ and different ``free'' tier Hamiltonians, but they live on a
common configuration space $\mathcal D_n$ \cite{stanley-catalan,cheneviere2022linear}.

\paragraph{Linear intervals as ``1D corridors'' in Catalan posets.}
A useful robustness fact is that certain one-dimensional substructures are
invariant across these Catalan posets: Chenevi\`ere proves that, for each fixed
$n$ and each height parameter $k$ (in the sense of \cite{cheneviere2022linear}),
the Tamari lattice and the Dyck lattice have the same number of \emph{linear
intervals} (intervals whose Hasse diagram is a chain), and moreover all
alt-Tamari posets share this same count at each height
\cite{cheneviere2022linear}. In the present language, this says that the number
of ``diamond-free corridors'' (regions with a unique maximal chain) is stable
under a wide class of tier-local adjacency choices, reinforcing the theme that
many distinct dynamics can be layered on a single Catalan substrate without
changing its most rigid combinatorial invariants.

\paragraph{Adjacency and Laplacian.}
Define the adjacency operator $A_{G_n}$ and the degree operator $D_{G_n}$ on
$\ell^2(\mathcal D_n)$ by
\[
(A_{G_n}\psi)(w) := \sum_{w'\sim w} \psi(w'),
\qquad
(D_{G_n}\psi)(w) := \deg(w)\,\psi(w),
\]
where $w'\sim w$ denotes adjacency in $G_n$. The (combinatorial) graph Laplacian is
\[
\Delta_{G_n} := D_{G_n}-A_{G_n},
\]
and the associated diffusion generator is
\[
L_{G_n} := -\Delta_{G_n}.
\]
Then $\Delta_{G_n}$ is self-adjoint and positive semidefinite, while $L_{G_n}$ is
self-adjoint and negative semidefinite.

\paragraph{Discrete heat and Schr\"odinger equations.}
The heat equation on the tier graph is the linear ODE
\[
\partial_\tau u = L_{G_n}u,
\]
with solution $u(\tau)=e^{\tau L_{G_n}}u(0)$. Because $-\Delta_{G_n}$ is
self-adjoint and nonpositive, $e^{\tau L_{G_n}}$ is a contraction semigroup.
The corresponding unitary ``free'' Schr\"odinger evolution is
\[
i\,\partial_t \psi = -L_{G_n}\psi = \Delta_{G_n}\psi,
\]
with solution $\psi(t)=e^{-it\Delta_{G_n}}\psi(0)$.

\begin{remark}
This optional tier-graph framework does not fix a preferred choice of adjacency
$G_n$ and is not used elsewhere in the paper. Its purpose is to make explicit
that, once a tier-local notion of neighbourhood is specified, discrete diffusion
and Schr\"odinger-type evolutions on the Catalan state space follow by standard
graph-Laplacian constructions (compare Remark~\ref{rem:dyck-conditioned-drift-scaling}
and Remark~\ref{rem:dyck-conditioned-kernel-doob} for the tier-growth Markov
structure induced by Dyck conditioning).
\end{remark}

% ---------------------------------------------------------------------------
% Bibitems used only in this supplement (kept here for copy/paste convenience).
% If you re-`\input{supplemental-operators.tex}` into the main paper, these
% entries should live in the main `thebibliography` environment.
% ---------------------------------------------------------------------------
\iffalse
\begin{thebibliography}{99}

\bibitem{abhy2018scatteringforms}
N.~Arkani-Hamed, Y.~Bai, S.~He, and G.~Yan.
\newblock Scattering forms and the positive geometry of kinematics, color and
the worldsheet.
\newblock {\em JHEP} \textbf{05} (2018) 096.
\newblock arXiv:1711.09102.

\bibitem{banerjee2018stokes}
P.~Banerjee, A.~Laddha, and P.~Raman.
\newblock Stokes polytopes: The positive geometry for $\phi^4$ interactions.
\newblock arXiv:1811.05904, 2018.

\bibitem{herrmann2022positivegeometry}
E.~Herrmann and J.~Trnka.
\newblock Positive geometry of scattering amplitudes.
\newblock arXiv:2203.13018, 2022.

\end{thebibliography}
\fi
` into the main paper, these
% entries should live in the main `thebibliography` environment.
% ---------------------------------------------------------------------------
\iffalse
\begin{thebibliography}{99}

\bibitem{abhy2018scatteringforms}
N.~Arkani-Hamed, Y.~Bai, S.~He, and G.~Yan.
\newblock Scattering forms and the positive geometry of kinematics, color and
the worldsheet.
\newblock {\em JHEP} \textbf{05} (2018) 096.
\newblock arXiv:1711.09102.

\bibitem{banerjee2018stokes}
P.~Banerjee, A.~Laddha, and P.~Raman.
\newblock Stokes polytopes: The positive geometry for $\phi^4$ interactions.
\newblock arXiv:1811.05904, 2018.

\bibitem{herrmann2022positivegeometry}
E.~Herrmann and J.~Trnka.
\newblock Positive geometry of scattering amplitudes.
\newblock arXiv:2203.13018, 2022.

\end{thebibliography}
\fi
` into the main paper, these
% entries should live in the main `thebibliography` environment.
% ---------------------------------------------------------------------------
\iffalse
\begin{thebibliography}{99}

\bibitem{abhy2018scatteringforms}
N.~Arkani-Hamed, Y.~Bai, S.~He, and G.~Yan.
\newblock Scattering forms and the positive geometry of kinematics, color and
the worldsheet.
\newblock {\em JHEP} \textbf{05} (2018) 096.
\newblock arXiv:1711.09102.

\bibitem{banerjee2018stokes}
P.~Banerjee, A.~Laddha, and P.~Raman.
\newblock Stokes polytopes: The positive geometry for $\phi^4$ interactions.
\newblock arXiv:1811.05904, 2018.

\bibitem{herrmann2022positivegeometry}
E.~Herrmann and J.~Trnka.
\newblock Positive geometry of scattering amplitudes.
\newblock arXiv:2203.13018, 2022.

\end{thebibliography}
\fi
` into the main paper, these
% entries should live in the main `thebibliography` environment.
% ---------------------------------------------------------------------------
\iffalse
\begin{thebibliography}{99}

\bibitem{abhy2018scatteringforms}
N.~Arkani-Hamed, Y.~Bai, S.~He, and G.~Yan.
\newblock Scattering forms and the positive geometry of kinematics, color and
the worldsheet.
\newblock {\em JHEP} \textbf{05} (2018) 096.
\newblock arXiv:1711.09102.

\bibitem{banerjee2018stokes}
P.~Banerjee, A.~Laddha, and P.~Raman.
\newblock Stokes polytopes: The positive geometry for $\phi^4$ interactions.
\newblock arXiv:1811.05904, 2018.

\bibitem{herrmann2022positivegeometry}
E.~Herrmann and J.~Trnka.
\newblock Positive geometry of scattering amplitudes.
\newblock arXiv:2203.13018, 2022.

\end{thebibliography}
\fi
